\chapter{集合与函数}
\makepart{单项选择题}



\begin{problem}
	用区间表示满足不等式 $|x| > |x - 4|$ 的所有 $x$ 的集合是\pickin{B}.
	
	
\begin{abcd} 
		\item $\left( - 2,2 \right)$
		
		\item $\left( 2, + \infty \right)$
		
		\item $\left( - \infty, - 2 \right)$
		
		\item $\left( - \infty, + \infty \right)$
\end{abcd}  
\end{problem}

\begin{problem}	
函数$\displaystyle f\left( x \right) = \frac{\sqrt{x^{2} - 4}}{x - 2}$ 的定义域是\pickin{C}.
	
	
	\begin{abcd} \item $\left( - \infty,2 \right) \cup \left( 2, + \infty \right)$
		
		\item
		$\left( - \infty, - 2 \right\rbrack \cup \left\lbrack 2, + \infty \right)$
		
		\item
		$\left( - \infty, - 2 \right\rbrack \cup \left( 2, + \infty \right)$
		
		\item
		$\left( - \infty, - 2 \right) \cup \left( - 2,2 \right) \cup \left( 2, + \infty \right)$
		
		
\end{abcd}  
\end{problem}

\begin{problem}
	函数
	$f\left( x \right) = \left\{ \begin{matrix}
	x - 3, & - 4 \leq x \leq 0 \\
	x^{2} + 1, & 0 < x \leq 3 \\
	\end{matrix} \right.\ $的定义域是\pickin{C}.
	
	
	\begin{abcd} \item $- 4 \leq x \leq 0$
		
		\item $0 \leq x \leq 3$
		
		\item $\left\lbrack - 4,3 \right\rbrack$
		
		\item
		$\left\{ \left. \ x \right| - 4 \leq x \leq 0 \right\} \cap \left\{ \left. \ x \right|0 < x \leq 3 \right\}$
		\end{abcd}
		
		\end{problem} 


		
		\begin{problem}
			设$f\left( x \right)$的定义域是$\left\lbrack 0,2 \right\rbrack$, 则$f\left( x^{2} \right)$的定义域是\pickin{D}.
			
			
			\begin{abcd} \item $\left\lbrack 0,4 \right\rbrack$
				
				\item $\left\lbrack 0,2 \right\rbrack$
				
				\item $\left\lbrack - 2,2 \right\rbrack$
				
				\item $\lbrack - \sqrt{2},\sqrt{2}\rbrack$
				\end{abcd}
				
		\end{problem}
				
				\begin{problem}
					下列各组中$f\left( x \right)$与$g\left( x \right)$是相同函数的是\pickin{C}.
					
					
					\begin{abcd} \item $f\left( x \right) = \sqrt{x^{2}}, g\left( x \right) = x$
						
						\item
						$f\left( x \right) = x + 1, g\left( x \right) = \frac{x^{2} - 1}{x - 1}$
						
						\item $f\left( x \right) = \ln x^{2}, g\left( x \right) = 2\ln|x|$
						
						\item $f\left( x \right) = \left\{ \begin{matrix}
						1,x \geq 0 \\
						-1,x < 0 \\
						\end{matrix} \right.\ ,g\left( x \right) = \frac{|x|}{x}$
						
				\end{abcd}  \end{problem}
				
				\begin{problem}
					设$f\left( x \right) = \left\{ \begin{matrix}
					x^{2}, &x \leq - 2 \\
					x + 9, &- 2 < x < 2 \\
					2^{x}, &x \geq 2 \\
					\end{matrix} \right.\ $, 则下列等式中不成立的是\pickin{B}.
					
					
					\begin{abcd} \item $f\left( - 2 \right) = f\left( 2 \right)$
						
						\item $f\left( 1 \right) = f\left( 4 \right)$
						
						\item $f\left( - 1 \right) = f\left( 3 \right)$
						
						\item $f\left( 0 \right) = f\left( - 3 \right)$
						
				\end{abcd}  \end{problem}
				
				\begin{problem}
					设 $y = f\left( x \right)$
					为单调增加函数, 则其反函数$y = f^{- 1}\left( x \right)$的单调性为\pickin{A}.
					
					
					\begin{abcd} \item 单调增加
						
						\item 单调减少
						
						\item 有增有减
						
						\item 不能确定
						
				\end{abcd}  \end{problem}
				
				\begin{problem}
					函数$f\left( x \right) = \arctan\frac{1}{x}$在其定义域上是\pickin{A}.
					
					
					\begin{abcd} \item 有界奇函数
						
						\item 有界偶函数
						
						\item 无界奇函数
						
						\item 无界偶函数
						
				\end{abcd}  \end{problem}
				
				\begin{problem}
					设
					$f\left( x \right) = x^{2} - 2$, $g\left( x \right) = 2x + 1$, 则复合函数
					$f\left\lbrack g\left( x \right) \right\rbrack =$ \pickin{B}.
					
					
					\begin{abcd} \item $4x^{2} + 4x + 3$
						
						\item $4x^{2} + 4x - 1$
						
						\item $2x^{2} - 3$
						
						\item $x^{2} + 2x + 1$
					\end{abcd}
				\end{problem}
				
				\begin{problem}
					下列函数必定是奇函数的是\pickin{C}.
					
					
					\begin{abcd} \item $y = f\left( x^{2} \right)$
						
						\item $y = \frac{1}{2}\left( \e^{x} + \e^{- x} \right)$
						
						\item $y = f\left( x \right) - f\left( - x \right)$
						
						\item $y = 5$
						
				\end{abcd}  \end{problem}
				
				\begin{problem}
					函数$y = 10^{x - 1} - 2$的反函数是\pickin{A}.
					
					
					\begin{abcd} \item $y = 1 + \lg\left( x + 2 \right)$
						
						\item $y = 1 + \lg\left( x - 2 \right)$
						
						\item $y = 1 + \ln\left( x + 2 \right)$
						
						\item $y = 1 - \lg\left( x + 2 \right)$
						
				\end{abcd}  \end{problem}
				
				\begin{problem}
					已知$f\left( x \right)$是线性函数, 且$f\left( - 1 \right) = 2$, $f\left( 1 \right) = - 2$, 则$f\left( x \right) =$
					\pickin{A}.
					
					
					\begin{abcd} \item $- 2x$
						
						\item $2x$
						
						\item $x - 3$
						
						\item $x + 3$
						
				\end{abcd}  \end{problem}
\begin{problem}$f\left( x \right) = x\left( \e^{x} - \e^{- x} \right)$在其定义域$\left( - \infty ,  + \infty \right)$内是\pickin{C}.
				
				\begin{abcd} \item 有界函数
					
					\item 单调增加函数
					
					\item 偶函数
					
					\item 奇函数
					
			\end{abcd}  \end{problem}
			
			\begin{problem}
				设$f(x)=p\sin x + 2qx\cos x + x^{2}$, 其中$p,q$为常数, 已知$f\left( 2 \right)\text{=3}$, 则$f( - 2 )=$\pickin{B}
				
				
				\begin{abcd} \item 3
					
					\item 5
					
					\item $p\sin 2 - 4q\cos 2 + 4$
					
					\item $8q\cos 2 + 5$
					
			\end{abcd}  \end{problem}
			
			\begin{problem}
				设$\displaystyle f\left( x \right) = \frac{x}{1 - x}$, $g\left( x \right) = 1 - x$, 则$f\left\lbrack g\left( x + 1 \right) \right\rbrack =$
				\pickin{A}.
				
				
				\begin{abcd} \item $\displaystyle \frac{- x}{1 + x}$
					
					\item $\displaystyle \frac{x}{1 + x}$
					
					\item $\displaystyle \frac{2x}{1 - x}$
					
					\item $\displaystyle \frac{1 + x}{x}$
					
			\end{abcd}  \end{problem}
			
			\begin{problem}
				下列函数中为奇函数的是\pickin{D}.
				\begin{abcd} \item $f\left( x \right) = \left\{ \begin{matrix}
					x,&\left| x \right| > 1 \\
					1,& 0 \leq x \leq 1 \\
					1,& - 1 < x < 0 \\
					\end{matrix} \right.\ ;$
					
					
					\item $\psi\left( x \right) = \left\{ \begin{matrix}
					-1, &- 1 < x < 0 \\
					1, &0 \leq x < 1 \\
					x,& \left| x \right| \geq 1 \\
					\end{matrix} \right.\ ;$
					
					\item $g\left( x \right) = \left\{ \displaystyle \begin{matrix}
					\e^{x}, &x \geq 0 \\
					-\frac{1}{\e^{x}},& x < 0 \\
					\end{matrix} \right.\ ;$ 
					
					\item
					$h\left( x \right) = \left\{ \begin{matrix}
					\e^{x},& x > 0 \\
					0,& x = 0 \\
					-\frac{1}{\e^{x}},& x < 0 \\
					\end{matrix} \right.\ .$
					
			\end{abcd}  \end{problem}
			
			\begin{problem}
				$f\left( x \right) = \left( \sin 3x \right)^{2}$在定义域$\left( - \infty, + \infty \right)$上为\pickin{B}.
				
				
				\begin{abcd} \item 周期是$\displaystyle 3\pi$的周期函数
					
					\item 周期是$\displaystyle \frac{\pi}{3}$的周期函数
					
					\item 周期是$\displaystyle \frac{2\pi}{3}$的周期函数
					
					\item 不是周期函数
					
			\end{abcd}  \end{problem}
			
			\begin{problem}
				函数$\displaystyle f\left( x \right) = \ln\frac{a - x}{a + x}\left( a > 0 \right)$是\pickin{A}.
				
				
				\begin{abcd} \item 奇函数
					
					\item 偶函数
					
					\item 非奇非偶函数
					
					\item 奇偶性决定于$a$的值
					
			\end{abcd}  \end{problem}
			
			\begin{problem}
				设$f\left( x \right) = \left\{ \begin{matrix}
				-x^{3}, & - 3 \leq x \leq 0 \\
				x^{3},  & 0 < x \leq 2 \\
				\end{matrix} \right.\ $, 则此函数是\pickin{C}.
				
				
				\begin{abcd} \item 奇函数
					
					\item 偶函数
					
					\item 有界函数
					
					\item 周期函数
					
			\end{abcd}  \end{problem}
			
			\begin{problem}下列函数中一定没有反函数的是\pickin{B}.
			
			\begin{abcd} \item 奇函数
				
				\item 偶函数
				
				\item 单调函数
				
				\item 有界函数
				
		\end{abcd}  \end{problem}
		
		\begin{problem}
			设$f\left( x \right) = x\left| x \right|,x \in \left( - \infty, + \infty \right)$, 则$f\left( x \right)$
			\pickin{B}.
			
			
			\begin{abcd} \item 在$\left( - \infty, + \infty \right)$单调减;
				
				\item 在$\left( - \infty, + \infty \right)$单调增;
				
				\item
				在$\left( - \infty,0 \right)$内单调增, 而在$\left( 0, + \infty \right)$内单调减;
				
				\item
				在$\left( - \infty,0 \right)$内单调减, 而在$\left( 0, + \infty \right)$内单调增.
				
		\end{abcd}  \end{problem}
		
		\begin{problem}
			设$f\left( x \right)$的定义域为$\left\lbrack 0,1 \right\rbrack$
			, 则函数$\displaystyle f\left( x + \frac{1}{4} \right) + f\left( x - \frac{1}{4} \right)$
			的定义域为\pickin{D}.
			
			
			\begin{abcd} \item $\displaystyle \left\lbrack 0,1 \right\rbrack$
				
				\item$\displaystyle \left\lbrack - \frac{1}{4},\frac{5}{4} \right\rbrack$
				
				\item $\displaystyle \left\lbrack - \frac{1}{4},\frac{1}{4} \right\rbrack$
				
				\item $\displaystyle \left\lbrack \frac{1}{4},\frac{3}{4} \right\rbrack$
				
		\end{abcd}  \end{problem}
		
		\begin{problem}
			函数$\displaystyle f\left( x \right) = \frac{2x}{1 + x^{2}}$在其定义域上是\pickin{A}.
			
			
			\begin{abcd} \item 有界奇函数
				
				\item 有界偶函数
				
				\item 无界奇函数
				
				\item 无界偶函数
				
		\end{abcd}  \end{problem}


\makepart{填空题}

\begin{problem}
	函数$f\left( x \right) = \arcsin\left( x^{2} - x - 1 \right)$的定义域$D =$\fillin{$\left\lbrack - 1, 0 \right\rbrack \cup \left\lbrack 1, 2 \right\rbrack$}.
\end{problem}


\begin{problem}
	函数$y = \text{lnln}x$的定义域$D =$\fillin{$\left( 1, + \infty \right)$} .
\end{problem} 

\begin{problem}
	函数$f\left( x \right) = \arcsin\left( x^{2} - x - 1 \right)$的定义域$D =$\fillin{$\left\lbrack - 1, 0 \right\rbrack \cup \left\lbrack 1, 2 \right\rbrack$}.
\end{problem}


\begin{problem}
	函数$y = \text{lnln}x$的定义域$D =$\fillin{$\left( 1, + \infty \right)$} .
\end{problem} 


\begin{problem}
	函数$\displaystyle y = \ln\sqrt[\uproot{20}\small 3]{\frac{1}{x} - 1}$的定义域$D =$\fillin{$\left( 0,1 \right)$}.
\end{problem} 

\begin{problem}
	设$f\left( x \right) = \left\{ \begin{matrix}
	\left| \sin x \right|, & \left| x \right| < 1 \\
	0\quad, & \left| x \right| \geq 1 \\
	\end{matrix} \right.\ $
	, 则$\displaystyle f\left( - \frac{\pi}{4} \right) =$\fillin{$\displaystyle \frac{\sqrt{2}}{2}$} .
\end{problem} 

\begin{problem}
	设$f\left( x \right) = \left\{ \begin{matrix}
	x + 3,  & 1 \leq x \leq 3 \\
	\cos 2,  & 3 < x \leq 5 \\
	\end{matrix} \right.\ $, 则$f\left( x + 2 \right)$的定义域为\fillin{$\left\lbrack - 1,3 \right\rbrack$}.
\end{problem} 



\begin{problem}设函数$f\left( x \right)$的定义域为$\left\lbrack - 1,1 \right\rbrack$, 则复合函数$f\left( \sin x \right)$的定义域为\fillin{$\left( - \infty, + \infty \right)$}.
	
\end{problem} 


\begin{problem}函数$\displaystyle f\left( x \right) = \frac{x}{1 + x}$的反函数$f^{- 1}\left( x \right) =$\fillin{$\displaystyle \frac{x}{1 - x}$}.
\end{problem} 

\begin{problem}设函数$f\left( x \right) = \e^{x}$, $g\left( x \right) = \sin x$, 则$f\left\lbrack g\left( x \right) \right\rbrack =$\fillin{$\e^{\sin x}$}.
\end{problem} 

\begin{problem}设$f\left( x \right) = \cos 2x$, $f\left\lbrack g\left( x \right) \right\rbrack = 1 - x^{2}$, 则$g\left( x \right) =$
	\fillin{$\displaystyle \frac{1}{2}\arccos\left( 1 - x^{2} \right)$} , $g\left( x \right)$的定义域为\fillin{$\left\lbrack - \sqrt{2},\sqrt{2} \right\rbrack$}.
\end{problem}

\begin{problem}$f\left( x \right) = \left\{ \begin{matrix}
	1 + x, & x < 2 \\
	x^{2} - 1, & x \geq 2 \\
	\end{matrix} \right.\ $的反函数$f^{- 1}\left( x \right) =$\fillin{$ \left\{ \begin{matrix}
		x - 1, & x < 3 \\
		\sqrt{x + 1}, & x \geq 3 \\
		\end{matrix} \right.\ $}.
\end{problem} 


\begin{problem}已知$f\left( x \right) = \sin x$, $f\left\lbrack \phi\left( x \right) \right\rbrack = 1 - x^{2}$, 则$\phi(x) = \arcsin\left( 1 - x^{2} \right)$的定义域为\fillin{$\lbrack - \sqrt{2},\sqrt{2}\rbrack$}.
\end{problem} 

\begin{problem}设$f\left( x + 1 \right) = \left\{ \begin{matrix}
	1 - x, & x \leq 0 \\
	1\quad, & x > 0 \\
	\end{matrix} \right.\ $, 则$f\left\lbrack f\left( x \right) \right\rbrack =$\fillin{1}.
\end{problem}

\begin{problem}若$\displaystyle f\left( x + \frac{1}{x} \right) = x^{2} + \frac{1}{x^{2}} + 3$, 则$f\left( x \right) =$\fillin{$x^{2} + 1$}.
\end{problem} 



\begin{problem}
	函数$y = \sqrt{3 - x} + \arcsin\frac{3 - 2x}{5}$的定义域为\fillin{$\left\lbrack - 1,3 \right\rbrack$}.
\end{problem} 

\begin{problem}
	设$\displaystyle f\left( \frac{1}{t} \right) = \frac{5}{t} + 2t^{2}$, 则$f\left( t^{2} + 1 \right) =$\fillin{$\displaystyle 5\left( t^{2} + 1 \right) + \frac{2}{\left( t^{2} + 1 \right)^{2}}$}.
\end{problem}


\begin{problem}设$f\left( x \right)$的定义域为$\left\lbrack 0,1 \right\rbrack$, 则$f\left( x + a \right) + f\left( x - a \right)$的定义域为\fillin{
		$ \left\{ \begin{matrix}
		\left\lbrack a,1 - a \right\rbrack, & 0 \leq a \leq \frac{1}{2}; \\
		\left\lbrack - a,1 + a \right\rbrack, & - \frac{1}{2} \leq a \leq 0 \\
		\end{matrix} \right.$}.
\end{problem}


\begin{problem}
	已知$f\left( x \right) = \arcsin x$, $\displaystyle g\left( x \right) = \frac{1}{\sqrt{x^{2} - 1}}$, 则$f\left\lbrack g\left( x \right) \right\rbrack$的定义域为\fillin{$\left\lbrack \left. \ \sqrt{2},  + \infty \right) \right.\  \cup \left( - \infty ,  \right.\ \left. \  - \sqrt{2} \right\rbrack$}.
\end{problem}


\begin{problem}
	若$\displaystyle f\left( x \right) = \frac{1}{1 - x}$, 则$f\left\{ f\left\lbrack f\left( x \right) \right\rbrack \right\} =$\fillin{$x$}.
\end{problem}

\begin{problem}
	设$f\left( x \right)$的定义域为$\left\lbrack 1,2 \right\rbrack$, 则$\displaystyle f\left( \frac{1}{x + 1} \right)$的定义域为 \fillin{$\displaystyle \left\lbrack - \frac{1}{2},0 \right\rbrack$}.
\end{problem}

\begin{problem}
	$f\left( x \right) = \log_{2}\left( \log_{2}x \right)$的定义域为\fillin{$\left( 1, + \infty \right)$}.
\end{problem} 


\makepart{计算题}

\begin{problem}
	设$f\left( x \right) = \left\{ \begin{matrix}
	x^{2} - x + 1, & x \leq 1 \\
	2x - x^{2}, & x > 1 \\
	\end{matrix} \right.\ $,求$f\left( 1 + a \right) + f\left( 1 - a \right)$,其中$a > 0$.
	
	
	
	
	\begin{solution} 因为$a > 0$,所以$ 1 + a > 1$,$1 - a < 1$,故
	$$f\left( 1 + a \right) = 2\left( 1 + a \right) - \left( 1 + a \right)^{2} = 1 - a^{2},$$
	$$f\left( 1 - a \right) = \left( 1 - a \right)^{2} - \left( 1 - a \right) + 1 = a^{2} - a + 1$$
	故$f\left( 1 + a \right) + f\left( 1 - a \right) = 2 - a.$
	\end{solution}
\end{problem}    


\begin{problem}
	设$f\left( x - 2 \right) = x^{2} - 2x + 3$,求$f\left( x + 2 \right)$.
	
\begin{solution}
	令$u = x - 2$,$x = u + 2$,代入得$$f\left( u \right) = \left( u + 2 \right)^{2} - 2\left( u + 2 \right) + 3 = u^{2} + 2u + 3$$	
	所以$$f\left( x + 2 \right) = \left( x + 2 \right)^{2} + 2\left( x + 2 \right) + 3 = x^{2} + 6x + 11.$$
\end{solution}	
\end{problem}    


\begin{problem}
	设$\displaystyle f\left( x + \frac{1}{x} \right) = \frac{x^{3} + x}{x^{4} + 3x^{2} + 1}\left( x \neq 0 \right)$,求$f\left( x \right)$.
	
	\begin{solution}由条件
	$$f\left( x + \frac{1}{x} \right) = \frac{x^{3} + x}{x^{4} + 3x^{2} + 1} = \frac{x + \frac{1}{x}}{x^{2} + \frac{1}{x^{2}} + 3} = \frac{x + \frac{1}{x}}{\left( x + \frac{1}{x} \right)^{2} + 1}$$
	所以
	$$f\left( x \right) = \frac{x}{x^{2} + 1}.$$
	\end{solution}
\end{problem}






\begin{problem}
	设$\displaystyle f\left( x \right) = \arcsin\frac{2x - 1}{5} + \sqrt{\sin\pi x}$,求$f\left( x \right)$的定义域.
	
	\begin{solution}
		由$\displaystyle \arcsin\frac{2x - 1}{5}$有$\displaystyle \left| \frac{2x - 1}{5} \right| \leq 1$,故$- 2 \leq x \leq 3$,又由$\sqrt{\sin\pi x}$有$\sin\pi x \geq 0$得
		$$2k \leq x \leq 2k + 1\quad\left( k = 0, \pm 1, \pm 2,\cdots \right)$$
		故函数的定义域为$\displaystyle \left\lbrack - 2, - 1 \right\rbrack \cup \left\lbrack 0,1 \right\rbrack \cup \left\lbrack 2,3 \right\rbrack$.
\end{solution}   
\end{problem}


\begin{problem}设$\displaystyle f\left( x \right) = \ln\frac{2 - x}{2 + x}$,求$\displaystyle f\left( x \right) + f\left( \frac{1}{x} \right)$的定义域.
	
	
	
	\begin{solution}
		由$$\ln\frac{2 - x}{2 + x},$$ 有$$\frac{2 - x}{2 + x} > 0,$$ 得$$- 2 < x < 2;$$
			
		当$x \neq 0$时,对$\displaystyle f\left( \frac{1}{x} \right)$有$\displaystyle x > \frac{1}{2}$或$\displaystyle x < - \frac{1}{2}$,
			故函数$\displaystyle f\left( x \right) + f\left( \frac{1}{x} \right)$的定义域为
			$$\left( - 2, - \frac{1}{2} \right) \cup \left( \frac{1}{2},2 \right)$$
	\end{solution}   \end{problem}
	\begin{problem}设$\displaystyle f\left( x \right) = \frac{1}{2}\left( x + \left| x \right| \right)$,$\varphi\left( x \right) = \left\{ \begin{matrix}
		x,x < 0 \\
		x^{2},x \geq 0 \\
		\end{matrix} \right.\ $,求$f\left\lbrack \varphi\left( x \right) \right\rbrack$.
		
		\begin{solution} $f\left( x \right) = \left\{ \begin{matrix}
			0,x < 0 \\
			x,x \geq 0 \\
			\end{matrix} \right.\ $,$\therefore\quad f\left\lbrack \varphi\left( x \right) \right\rbrack = \left\{ \begin{matrix}
			0,x < 0 \\
			x^{2},x \geq 0 \\
			\end{matrix} \right.\ $.
	\end{solution}   \end{problem}
	\begin{problem}求函数$\displaystyle y = \ln\frac{a - x}{a + x}\left( a > 0 \right)$的反函数的形式.
		
		\begin{solution} 由$$\frac{a - x}{a + x} = \e^{y}$$
			得$$x = \frac{a\left( 1 - \e^{y} \right)}{1 + \e^{y}},$$所求反函数为 $$y = \frac{a\left( 1 - \e^{x} \right)}{1 + \e^{x}}.$$
\end{solution}
\end{problem}
			
\begin{problem}$f\left( x \right) = \sin x$,$f\left\lbrack \varphi\left( x \right) \right\rbrack = 1 - x^{2}$,求$\varphi\left( x \right)$及其定义域.			

\begin{solution}
因为
$ f\left\lbrack \varphi\left( x \right) \right\rbrack = 1 - x^{2}\text{=sin}\varphi\left( x \right)$,
所以
$$\varphi\left( x \right) = \text{arcsin}\left( 1 - x^{2} \right)$$
故$\left| 1 - x^{2} \right| \leq 1$ $\Rightarrow$ 定义域为
$\left\lbrack - \sqrt{2},\sqrt{2} \right\rbrack$.
\end{solution}   
\end{problem}
		
		
\begin{problem}设$f\left( x \right) = \left\{ \begin{matrix}
				-\e^{x}, \ x \leq 0 \\
				x, \ x > 0 \\
				\end{matrix} \right.\ $,$\varphi\left( x \right) = \left\{ \begin{matrix}
				0, x \leq 0 \\
				-x^{2}, x > 0 \\
				\end{matrix} \right.\ $,求$f\left( x \right)$的反函数$g\left( x \right)$及$f\left\lbrack \varphi\left( x \right) \right\rbrack$.
				
				\begin{solution} $f\left( x \right)$的反函数
					$$g\left( x \right) = \left\{ \begin{matrix}
					\ln\left( - x \right), - 1 \leq x < 0 \\
					x\quad,x > 0 \\
					\end{matrix} \right. , $$
					从
					而
					$$f\left\lbrack \varphi\left( x \right) \right\rbrack = \left\{ \begin{matrix}
					-1, x \leq 0 \\
					-\e^{- x^{2}}, x > 0 \\
					\end{matrix} \right. .$$
			\end{solution}   
\end{problem}
		
\begin{problem}设$f\left( x \right) = \left\{ \begin{matrix}
				\e^{x},\ - \infty < x < 0 \\
				\sqrt{x} + 1, \ 0 \leq x \leq 4 \\
				x - 1,\ 4 < x < + \infty \\
				\end{matrix} \right.\ $,求$f\left( x \right)$的反函数$\varphi\left( x \right)$.
				
\begin{solution}由条件知
	\begin{enumerate}
		\item 当$- \infty < x < 0$时,$y = \e^{x}$,即$x = \ln y$,$0 < y < 1$;
		\item 当$0 \leq x \leq 4$时,$y = \sqrt{x} + 1$,即$x = \left( y - 1 \right)^{2}$,$1 \leq y \leq 3$;
		\item 当$4 < x < + \infty$时,$y = x - 1$,即$x = y + 1$,$y > 3$;
	\end{enumerate}
	
	故得反函数 $$\varphi\left( x \right) = \left\{ \begin{matrix}
	\ln x,  & 0 < x < 1 \\
	\left( x - 1 \right)^{2}, & 1 \leq x \leq 3 \\
	x + 1, & 3 < x < + \infty \\
	\end{matrix} \right. .$$		
\end{solution}   
\end{problem}


\makepart{综合与应用题}

\begin{problem}设$y = 1 + a + f\left( \sqrt{x} - 1 \right)$满足条件$\left. \ y \right|_{a = 0} = x$及$\left. \ y \right|_{x = 1} = 2$,求$f\left( x \right)$及$y$.
	
	\begin{solution} 由$\left. \ y \right|_{a = 0} = x$得
		$$f\left( \sqrt{x} - 1 \right) = x - 1 = \left( \sqrt{x} - 1 \right)^{2} + 2\left( \sqrt{x} - 1 \right)$$
		故$f\left( x \right) = x^{2} + 2x$,此时
		$$y = 1 + a + x - 1 = a + x$$
		又由$\left. \ y \right|_{x = 1} = 2$得$a = 1$,故$y = 1 + x$.
	\end{solution}   
\end{problem}

\begin{problem}设$\displaystyle f\left( x \right) = \frac{\sqrt{9 - x^{2}}}{\ln\left( x + 2 \right)} + \arcsin\frac{2x - 1}{4}$,求$f\left( x \right)$的定义域.
	
	\begin{solution} 由$$9 - x^{2} \geq 0$$得$$- 3 \leq x \leq 3;$$
		
		由$$x + 2 > 0 \text{ 且 } x + 2 \neq 1,$$ 得$$\ x > - 2 \text{ 且 }x \neq - 1;$$
		
		由$$\left| \frac{2x - 1}{4} \right| \leq 1$$得$$- \frac{3}{2} \leq x \leq \frac{5}{2};$$
		
		故函数的定义域为$$\left. \ \left\lbrack - \frac{3}{2} \right.\ , - 1 \right) \cup \left( \left. \  - 1,\frac{5}{2} \right\rbrack \right.\ .$$
	\end{solution}   
\end{problem}   


\begin{problem}已知$f\left( x \right) = \e^{x^{2}}$,$f\left\lbrack \varphi\left( x \right) \right\rbrack = 1 - x$,且$\varphi\left( x \right) \geq 0$,求$\varphi\left( x \right)$并写出它的定义域.
	
	\begin{solution}
		易知
		$$f\left\lbrack \varphi\left( x \right) \right\rbrack = e^{\left\lbrack \varphi\left( x \right) \right\rbrack^{2}},$$
		
		由$e^{\left\lbrack \varphi\left( x \right) \right\rbrack^{2}} = 1 - x$及$\varphi\left( x \right) \geq 0$得
		$\varphi\left( x \right) = \sqrt{\ln\left( 1 - x \right)}$,定义域为$x \leq 0$.
	\end{solution}    
\end{problem}      


\begin{problem}求函数$\displaystyle y = x\left| x \right| + 4x$的反函数.
	
	\begin{solution} 由$y = x\left( \left| x \right| + 4 \right)$得,$y$与$x$同号.
		
		(1) 当$x \geq 0$时,$x^{2} + 4x - y = 0$,得$x = - 2 \pm \sqrt{4 + y}\quad\left( x \geq 0 \right)$,故 $$x = - 2 + \sqrt{4 + y},$$
		
		(2) 当$x < 0$时,$x^{2} - 4x + y = 0$,得$x = 2 \pm \sqrt{4 - y}\quad\left( x < 0 \right)$,故 $x = 2 - \sqrt{4 - y}$.
		
		故反函数为 $$y = \left\{ \begin{matrix}
		2 - \sqrt{4 - x}, x < 0 \\-2 + \sqrt{4 + x}, x \geq 0 \\
		\end{matrix} \right.$$
	\end{solution}   
\end{problem}  



\begin{problem}判定函数$f\left( x \right) = \left( e^{x + \left| x \right|} - 1 \right) \cdot \ln\left( 1 + \left| x \right| - x \right)$的奇偶性.
	
	\begin{solution}
		由条件知
		\begin{enumerate}
			\item 当$x \geq 0$时,$\left| x \right| - x = 0$,得$\ln\left( 1 + \left| x \right| - x \right) = 0$,
			从而 $f\left( x \right) = 0$;
			\item 当$x < 0$时,$\left| x \right| + x = 0$,得$e^{x + \left| x \right|} - 1 = 0$,从而 $f\left( x \right) = 0$;
		\end{enumerate}
		
		
		
		
		综上述,对任意$x$,$f\left( x \right) \equiv 0$,
		故
		$f\left( - x \right) = 0 = f\left( x \right)$,$f\left( - x \right) = 0 = - f\left( x \right)$,即
		$f\left( x \right)$既是奇函数又是偶函数.
		
	\end{solution}   
	
\end{problem}           



\begin{problem}设$f\left( x \right)$对一切实数$x_{1}$,$x_{2}$成立
	$f\left( x_{1} + x_{2} \right) = f\left( x_{1} \right)f\left( x_{2} \right)$,
	且$f\left( 0 \right) \neq 0$,$f\left( 1 \right) = a$,求$f\left( 0 \right)$及$f\left( n \right)$.
	($n$为正整数).
	
	\begin{solution} 
		取$x_{1} = x_{2} = 0$代入已知式,得		
		$$f\left( 0 + 0 \right) = f\left( 0 \right) \cdot f\left( 0 \right),\ f\left( 0 \right) \neq 0,$$
		因此$ f\left( 0 \right) = 1$.
		又
		$$f\left( 1 \right) = a,\ f\left( 2 \right) = f\left( 1 + 1 \right) = f\left( 1 \right)f\left( 1 \right) = a^{2},$$
		
		设 $f\left( k \right) = a^{k}$, 则
		$$f\left( k + 1 \right) = f\left( k \right) \cdot f\left( 1 \right) = a^{k} \cdot a = a^{k + 1}$$
		
		故对一切$n$,有$f\left( n \right) = a^{n}.$
		
	\end{solution}   
\end{problem}



\begin{problem}某厂按年度计划消耗某种零件48000件,若每个零件每月库存费0.02元,采购费每次160元,为节省库存费,分批采购.
	试将全年总的采购费和库存费这两部分的和$f\left( x \right)$表示为批量$x$的函数.
	
	\begin{solution} 		
		易知
		\begin{equation*}
		\begin{split}
		f\left( x \right) &=\text{库存费}+\text{采购费}\\
		&= 0.02 \times 12 \times \frac{x}{2} + \frac{48000}{x} \times 160 = 0.12x + \frac{7.68 \times 10^{6}}{x}
		\end{split}
		\end{equation*}
	\end{solution}   
\end{problem}    

\begin{problem}
	市场中某种商品的需求函数为$q_{d} = 25 - p$,而该种商品的供给函数为
	$\displaystyle q_{s} = \frac{20}{3}p - \frac{40}{3},$
	试求市场均衡价格和市场均衡数量.
	
	\begin{solution} 由均衡条件$q_{d} = q_{s}$得
		$$25 - p = \frac{20}{3}p - \frac{40}{3},$$
		移项整理得 $$23p = 115\quad \Rightarrow \quad p_{0} = 5,$$
		$$q_{0} = \frac{20}{3}p_{0} - \frac{40}{3}\quad \Rightarrow \quad q_{0} = 20,$$
		即市场均衡价格为5,市场均衡数量为20.
		
\end{solution}  
\end{problem} 



\begin{problem}某商品的成本函数(单位:元)为
	$C = 81 + 3q,$ 其中$q$为该商品的数量. 试问:
	
	(1) 如果商品的售价为12元/件,该商品的保本点是多少?
	
	(2) 售价为12元/件时,售出10件商品时的利润为多少?
	
	(3) 该商品的售价为什么不应定为2元/件?
	
	\begin{solution}
		(1)依题意,$C\left( q \right) = 81 + 3q = 12q = R\left( q \right)\Rightarrow \quad q = 9$ (件);
		
		(2)
		$L\left( 10 \right) = R\left( 10 \right) - C\left( 10 \right) = 12 \times 10 - 81 - 3 \times 10 = 9$
		(元);
		
		(3)若商品的售价实为2元/件,则
		$$L\left( q \right) = R\left( q \right) - C\left( q \right) = 2q - \left( 81 + 3q \right) = - 81 - q < 0\ \text{(元)}$$		
		从而无法盈利.
		
	\end{solution}   
\end{problem} 

\begin{problem}某商品的需求量$Q$是价格$P$的线性函数$Q = a + bP$,已知该商品的最大需求量为40000件(价格为零时的需求量),最高价格为40元/件(需求量为零时的价格).
	求该商品的需求函数与收益函数.
	
	\begin{solution} 
		因为
		$Q = a + bP,$
		
		当$P = 0$时,将$Q = 40000$代入上式得$a = 40000$;
		
		当$Q = 0$时,将$P = 40$代入得$a + 40b = 0$,故$b = - 1000$;
		
		需求函数: $Q = 40000 - 1000P$
		
		收益函数:$R\left( Q \right) = P \cdot Q = Q\left( \displaystyle \frac{40000 - Q}{1000} \right) = 40Q - \displaystyle \frac{Q^{2}}{1000}.$
		
	\end{solution}   
\end{problem}

\begin{problem}收音机每台售价为90元,成本为60元.
	厂方为鼓励销售商大量采购,决定凡是订购量超过100台以上的,每多订购1台,售价就降低1分,但最低价为每台75元.
	
	(1) 将每台的实际售价$p$表示为订购量$x$的函数;
	
	(2) 将厂方所获的利润$l$表示为订购量$x$的函数;
	
	(3) 某一商行订购了1000台,厂方可获利润多少?
	
	\begin{solution} (1)依题意,得$p = \left\{ \begin{matrix}
		90,& x \leq 100 \\
		90 - 0.01\left( x - 100 \right),& 100 < x \leq 1600 \\
		75,& x > 1600 \\
		\end{matrix} \right.\ $;
		
		(2) 由(1)及已知条件,得$l = \left\{ \begin{matrix}
		30x,& x \leq 100 \\
		\left( 31 - 0.01x \right)x, & 100 < x \leq 1600 \\
		15x,& x > 1600 \\
		\end{matrix} \right.\ $;
		
		(3)
		当$x = 1000$时,$l = \left( 31 - 0.01 \times 1000 \right) \times 1000 = 21000$
		(元).
		
	\end{solution}   
\end{problem}


\makepart{分析与证明题}

\begin{problem}
	证明$f\left( x \right) = \left( 2 + \sqrt{3} \right)^{x} - \left( 2 - \sqrt{3} \right)^{x}$是奇函数.

\begin{solution}
	因$\left( 2 + \sqrt{3} \right)\left( 2 - \sqrt{3} \right) = 1$,即$\dfrac{1}{2 + \sqrt{3}} = 2 - \sqrt{3}$,于是
$${f\left( - x \right) = \left( 2 + \sqrt{3} \right)^{- x} - \left( 2 - \sqrt{3} \right)^{- x}
}{ = - \left\lbrack \left( 2 + \sqrt{3} \right)^{x} - \left( 2 - \sqrt{3} \right)^{x} \right\rbrack
}{ = - f\left( x \right)}$$
故$f\left( x \right)$是奇函数.
\end{solution}


\end{problem}

\begin{problem}
	设函数$y = f\left( x \right),x \in \left( - \infty, + \infty \right)$的图形关于$x = a$,$x = b$均对称$\left( a \neq b \right)$,试证:$y = f\left( x \right)$是周期函数,并求其周期.

\begin{solution}
	依题设$f\left( a + x \right) = f\left( a - x \right)$,$f\left( b + x \right) = f\left( b - x \right)$,于是,
$$\begin{aligned}
f\left( x \right) &= f\left\lbrack a + \left( x - a \right) \right\rbrack = f\left\lbrack a - \left( x - a \right) \right\rbrack
\\
 &= f\left( 2a - x \right) = f\left\lbrack b + \left( 2a - x - b \right) \right\rbrack
\\
& = f\left\lbrack b - \left( 2a - x - b \right) \right\rbrack
\\
&= f\left\lbrack x + 2\left( b - a \right) \right\rbrack
\end{aligned}$$

故$f\left( x \right)$是周期函数,其周期$T = 2\left( b - a \right)$.
\end{solution}
\end{problem}

