\chapter{不定积分}


\makepart{单项选择题}

\begin{problem} 设$f(x)$是$g(x)$的原函数, 则下列各式中正确的是 \pickin{B}.

\begin{abcd} \item$\int f(x)\dx = g(x) + C;$

\item $\int g(x)\dx = f(x) + C;$

\item $\int f'(x)\dx = g(x) + C;$

\item $\int g'(x)\dx = f(x) + C.$

\end{abcd}

\end{problem}           

\begin{problem} 下列各式中等于$f(x)$的是 \pickin{D}.

\begin{abcd} \item$\int \d f(x);$

\item $\d\int f(x)\dx;$

\item $\int f'(x)\dx;$

\item $\left( \int f(x)\dx \right)'.$

\end{abcd}

\end{problem}           

\begin{problem} $\int f(x)\dx = \sqrt{2x^{2} + 1} + C$则
$\int xf\left( 2x^{2} + 1 \right)\dx =$ \pickin{D}.

\begin{abcd} 
\item$\displaystyle x\sqrt{2x^{2} + 1} + C;$

\item $\displaystyle \frac{1}{2}\sqrt{2x^{2} + 1} + C;$

\item $\displaystyle \frac{1}{4}\sqrt{2x^{2} + 1} + C;$

\item $\displaystyle \frac{1}{4}\sqrt{2(2x + 1)^{2} + 1} + C.$

\end{abcd}

\end{problem}           

\begin{problem} 函数 $\displaystyle \cos\frac{\pi}{2}x$ 的一个原函数是 \pickin{A}.

\begin{abcd} 
	
\item$\displaystyle \frac{2}{\pi}\sin\frac{\pi}{2}x;$

\item $\displaystyle \frac{\pi}{2}\sin\frac{\pi}{2}x;$

\item $\displaystyle - \frac{2}{\pi}\sin\frac{\pi}{2}x;$

\item $\displaystyle - \frac{\pi}{2}\sin\frac{\pi}{2}{x.}$

\end{abcd}

\end{problem}           

\begin{problem} $\int 3^{x}\e^{x}\dx =$ \pickin{D}.

\begin{abcd} 
\item$\displaystyle (3\e)^{x} + C;$

\item $\displaystyle \frac{1}{3}(3\e)^{x} + C;$

\item $\displaystyle 3\e^{x} + C;$

\item $\displaystyle \frac{(3\e)^{x}}{1 + \ln 3} + C.$

\end{abcd}

\end{problem}           

\begin{problem} $\displaystyle \int \frac{{\dx}}{\sqrt{1 - 2x}} =$ \pickin{B}.

\begin{abcd} 
	
\item$\sqrt{1 - 2x} + C;$

\item $- \sqrt{1 - 2x} + C;$

\item $- \frac{1}{2}\sqrt{1 - 2x} + C;$

\item $- 2\sqrt{1 - 2x} + C.$

\end{abcd}

\end{problem}           

\begin{problem} 设$\displaystyle \int \frac{x}{f(x)}\dx = \ln(1 + x) + C$,
则$\displaystyle \int \frac{f(x)}{x}\dx =$ \pickin{D}.

\begin{abcd} 
\item $\displaystyle \frac{1}{\ln(1 + x)} + C; $

\item $\displaystyle \frac{\ln(1 + x)}{x} + C; $

\item $\displaystyle \frac{x^{2}}{2} + \frac{x^{3}}{3} + C; $

\item $\displaystyle x + \frac{x^{2}}{2} + C.$

\end{abcd}

\end{problem}           

\begin{problem} 不定积分 $\displaystyle \int \sin^2 {\frac{x}{2}} =$ \pickin{C}.

\begin{abcd} 
\item $\displaystyle 2{\cos^{2}}\frac{x}{2} + C; $

\item $\displaystyle x + \sin x + C; $

\item $\displaystyle \frac{1}{2}(x - \sin x) + C; $

\item $\displaystyle {1 - 2}{\sin^{2}}\frac{x}{2} + C.$

\end{abcd}

\end{problem}           

\begin{problem} 
	$\displaystyle \int \frac{1}{(\arcsin x)^{2}\sqrt{1 - x^{2}}}\dx =$ \pickin{B}.

\begin{abcd} 
	
\item $\dfrac{2}{3}\left( 1 - x^{2} \right)^{3/2} + C; $

\item $- \dfrac{1}{\arcsin x} + C; $

\item $\pm \dfrac{1}{\arcsin x} + C; $

\item $- \dfrac{2}{3}\left( 1 - x^{2} \right)^{3/2} + C.$

\end{abcd}

\end{problem}           

\begin{problem} $\displaystyle \int x^{5}\e^{x^{3}}\dx =$ \pickin{B}.

\begin{abcd} 
	
\item $\displaystyle \frac{1}{3}\e^{x}(x - 1) + C; $

\item $\displaystyle \frac{1}{3}\e^{x^{3}}\left( x^{3} - 1 \right) + C; $

\item $\displaystyle \e^{x^{3}}\left( x^{3} - 1 \right) + C; $

\item $\displaystyle \e^{x^{3}}\left( x^{3} + 1 \right) + C.$

\end{abcd}

\end{problem}           

\begin{problem} $f(x)$的一个原函数为$\ln x$, 则$f'(x) =$ \pickin{C}.

\begin{abcd} \item$1/x$;

\item $x\ln x - x + C; $

\item $- 1/x^{2}; $

\item $\e^{x}.$

\end{abcd}

\end{problem}           

\begin{problem} $x^{x}(1 + \ln x)$的原函数是 \pickin{B}.

\begin{abcd} 
	
\item $\displaystyle \frac{1}{1 + x}x^{x + 1} + \ln x + C; $

\item $x^{x} + C; $

\item $x\ln x + C; $

\item $\displaystyle \frac{1}{2}x^{x}\ln x + C.$

\end{abcd}

\end{problem}           

\begin{problem} 当$x < - 1$时, $\displaystyle \int \frac{1}{x\sqrt{x^{2} - 1}}{\dx} =$ \pickin{B}.

\begin{abcd} 
\item $\displaystyle \frac{1}{2}\sqrt{x^{2} - 1} + C; $

\item $\displaystyle \arcsin\frac{1}{x} + C; $

\item $\displaystyle - \arcsin\frac{1}{x} + C; $

\item $\displaystyle \pm \arcsin\frac{1}{x} + C.$

\end{abcd}

\end{problem}           

\begin{problem} $\int x^{2}\sin 2x\dx =$ \pickin{B}.

\begin{abcd} 
	\item $\displaystyle \frac{x}{2}\left( \frac{x}{2}\cos x + \sin 2x \right) + C; $

\item $\displaystyle \frac{1 - 2x^{2}}{4}\cos 2x + \frac{x}{2}\sin 2x + C; $

\item $\displaystyle \frac{1 - x^{2}}{4}(\cos 2x + \sin 2x) + C; $

\item $\displaystyle \frac{1 - x^{2}}{4}\cos 2x + \frac{x}{2}\sin 2x + C.$

\end{abcd}

\end{problem}           

\begin{problem} $\int {(\arcsin x)^{2}}\dx =$ \pickin{C}.

\begin{abcd} 
	\item $\displaystyle x(\arcsin x)^{2} + C; $

\item $\displaystyle x(\arcsin x)^{2} + \frac{\arcsin x}{\sqrt{1 - x^{2}}} + C; $

\item $\displaystyle x(\arcsin x)^{2} + 2\sqrt{1 - x^{2}}\arcsin x - 2x + C;$

\item
$\displaystyle x(\arcsin x)^{2} + \frac{2\arcsin x}{3\left( 1 - x^{2} \right)^{3}} + C.$

\end{abcd}

\end{problem}           

\begin{problem} $\displaystyle \int \frac{1}{1 + \cos x}\dx =$ \pickin{C}.

\begin{abcd} 
	
	\item $\displaystyle \tan x - \sec x + C; $

\item $\displaystyle \cot x - \csc x + C;$

\item $\displaystyle \tan\frac{x}{2} + C; $

\item $\displaystyle \tan\left( \frac{x}{2} - \frac{\pi}{4} \right) + C.$

\end{abcd}

\end{problem}           

\begin{problem} $\displaystyle \int \frac{\sin x\cos x }{\sin^4 x + \cos^4 x}\dx=$\pickin{B}.

\begin{abcd} \item $\displaystyle \frac{1}{2}\arctan(\cos 2x) + C;$

\item $\displaystyle - \frac{1}{2}\arctan(\cos 2x) + C; $

\item $\displaystyle \arctan( - \cos 2x) + C;$

\item
$\displaystyle \frac{1}{2}\ln\left| \frac{\sin 2x - 1}{\sin 2x + 1} \right| + C.$

\end{abcd}

\end{problem}           \begin{problem} 设$\displaystyle I = \int \frac{{\dx}}{1 + \sqrt{x}}$, 则 $I =$ \pickin{C}.

\begin{abcd} \item $\displaystyle - 2\sqrt{x} + 2\ln(1 + \sqrt{x}) + C;$

\item $\displaystyle 2\sqrt{x} + 2\ln(1 + \sqrt{x}) + C;$

\item $\displaystyle 2\sqrt{x} - 2\ln(1 + \sqrt{x}) + C;$

\item $\displaystyle - 2\sqrt{x} - 2\ln(1 + \sqrt{x}) + C.$

\end{abcd}

\end{problem}           

\begin{problem} $\displaystyle \int \sqrt{\frac{1 + x}{1 - x}}\dx =$ \pickin{B}.

\begin{abcd} \item $\displaystyle x - \cos x - C;$

\item $\displaystyle {\arcsin}x - \sqrt{1 - x^{2}} + C;$

\item $\displaystyle \arcsin x + \sqrt{1 - x^{2}} + C;$

\item $\displaystyle {\arccos}x - \sqrt{1 - x^{2}} + C.$

\end{abcd}

\end{problem}           

\begin{problem}
$\displaystyle \int {\frac{x\ln(x + \sqrt{1 + x^{2}})}{(1 + x^{2})^{2}}\dx =}$ \pickin{D}.

\begin{abcd} \item $\displaystyle \frac{1}{1 + x^{2}}\ln(x + \sqrt{1 + x^{2}}) + C;$

\item $\displaystyle \frac{\ln(x + \sqrt{1 + x^{2}})}{4(1 + x^{2})^{2}} + C;$

\item $\displaystyle - \frac{1}{2}\frac{1}{1 + x^{2}}\ln(x + \sqrt{1 + x^{2}}) + C;$ 
\item
$\displaystyle \frac{x}{2\sqrt{1 + x^{2}}} - \frac{1}{2(1 + x^{2})}\ln(x + \sqrt{1 + x^{2}}) + C.$

\end{abcd}

\end{problem}           

\begin{problem}
将$\displaystyle \frac{x + 1}{x^{2}(x^{2} + 1)(x^{2} + x + 1)}$分解为部分分式, 下列做法中, 正确的做法是设它为\pickin{D}.

\begin{abcd} 
	
\item $\displaystyle \frac{a}{x^{2}} + \frac{b}{1 + x^{2}} + \frac{c}{x^{2} + x + 1};$

\item
$\displaystyle \frac{a}{x^{2}} + \frac{b}{1 + x^{2}} + \frac{c_{1}x + c_{2}}{x^{2} - x + 1};$

\item
$\displaystyle \frac{a}{x} + \frac{b}{x^{2}} + \frac{c}{1 + x^{2}} + \frac{d}{x^{2} + x + 1};$

\item
$\displaystyle \frac{a_{1}}{x} + \frac{a_{2}}{x^{2}} + \frac{b_{1}x + b_{2}}{1 + x^{2}} + \frac{c_{1}x + c_{2}}{x^{2} + x + 1}$

\end{abcd}

\end{problem}   

\begin{problem} $\displaystyle \int \frac{\sin^2 x}{\sin^2 x +1}= $ \pickin{B}.

\begin{abcd} \item $\displaystyle \ln|\sin^2 x + 1|+ C$;

\item $\displaystyle x - \frac{1}{\sqrt{2}}{\arctan}(\sqrt{2}\tan x) + C;$

\item $\displaystyle x - \arctan(\sqrt{2}x) + C;$

\item $\displaystyle x - \arctan\left( \frac{\tan x}{\sqrt{2}} \right) + C.$

\end{abcd}

\end{problem}           

\begin{problem} $I = \int {\e^{2x}\sin 3x\dx =}$ \pickin{D}.

\begin{abcd} \item $\displaystyle \frac{\e^{2x}}{13}(3\sin 3x - 2\cos 2x) + C;$

\item $\displaystyle \frac{\e^{2x}}{13}(3\sin 3x + 2\cos 2x) + C;$

\item $\displaystyle \frac{\e^{2x}}{5}(2\sin 3x - 3\cos 3x) + C;$

\item $\displaystyle \frac{\e^{2x}}{13}(2\sin 3x - 3\cos 3x) + C.$

\end{abcd}

\end{problem}           


\begin{problem}
已知函数$F(x)$的导数为$\displaystyle f(x) = \frac{1}{\sin^{2}x + 2\cos^{2}x}$, 且$\displaystyle F(\frac{\pi}{4}) = 0$, 则$F(x) =$
\pickin{B}.

\begin{abcd} 
	\item $\displaystyle \ln\left| 1 + \cos^{2}x \right| - \ln\frac{3}{2};$

\item
$\displaystyle \frac{1}{\sqrt{2}}\arctan\frac{\tan x}{\sqrt{2}} - \frac{1}{\sqrt{2}}\arctan\frac{1}{\sqrt{2}};$

\item
$\displaystyle \frac{1}{2\sqrt{2}}\ln\left| \frac{\sqrt{2} - \sin x}{\sqrt{2} + \sin x} \right|; $

\item
$\displaystyle \frac{1}{2\sqrt{2}}\ln\left| \frac{\sqrt{2} - \sin x}{\sqrt{2} + \sin x} \right| - \frac{1}{2\sqrt{2}}\ln\left| 3 - 2\sqrt{2} \right|.$

\end{abcd}

\end{problem}   


\begin{problem} 设$f(x) \neq 0$, 且有连续的二阶导数, 则
$\displaystyle \int\left\{ \frac{f'(x)}{f(x)} - \frac{\left( f'(x) \right)^{2}}{(f(x))^{2}} \right\} \dx = $ \pickin{A}.

\begin{abcd} \item $\displaystyle \frac{f'(x)}{f(x)} + C;$ \item $\displaystyle \frac{f(x)}{f'(x)} + C;$ \item
$\displaystyle f(x)f'(x) + C;$ \item $\displaystyle \left\lbrack f'(x) \right\rbrack^{2} + C.$

\end{abcd}

\end{problem}

\makepart{填空题}

\begin{problem} 设$f\left( x \right)\dx = F\left( x \right) + C$, 则
$\int\sin x f(\cos x){\d}x$ \fillin{ $- F(\cos x) + C$}.

\end{problem}           

\begin{problem} 设$\int f(x)\dx = F(x) + C$, 则
$\int f(\sin x)\cos x\dx =$ 
\fillin{ $F(\sin x) + C$}.

\end{problem}           

\begin{problem} 设$\int f(x)\dx = F(x) + C$, 则 $\int xf'(x)\dx =$
\fillin{ $xf(x) - F(x) + C$}.

\end{problem}           

\begin{problem} 如果等式
$\displaystyle \int f(x)\e^{- \frac{1}{x}}dx = - \e^{\frac{1}{x}} + C$ 成立,
则函数 $f(x) =$ \fillin{ $\displaystyle \frac{1}{x^{2}}\e^{\frac{2}{x}}$}.

\end{problem}           

\begin{problem} 设 $\int {xf(x)\dx} = \arcsin x + C,$ 则
$\displaystyle \int \frac{1}{f(x)}\dx =$ \fillin{ $\displaystyle - \frac{1}{3}\left( 1 - x^{2} \right)^{3/2} + C$}.

\end{problem}           

\begin{problem}
若$\int {f(x)\dx = F(x) + C}$, 则$\int {\e^{- x}f(\e^{- x})\dx =}$
\fillin{ $- F\left( \e^{- x} \right) + C$}.

\end{problem}           


\begin{problem}
$\int \left( \sin\frac{x}{2} - \cos\frac{x}{2} \right)^{2}\dx =$
\fillin{ $x + \cos x + C$}.

\end{problem}           

\begin{problem} 若$\e^{- x}$是 $f(x)$ 的一个原函数, 则 $\int {xf(x)\dx} =$
\fillin{ $(x + 1)\e^{- x} + C$}.

\end{problem}           

\begin{problem} 若$f(x) = \e^{- x}$, 则 $\displaystyle \int \frac{f'(\ln x)}{x}\dx =$
\fillin{ $\displaystyle \frac{1}{x} + C$}.

\end{problem}           

\begin{problem} 若$\int f(x)\dx = x^{2} + C$, 则
$\int xf\left( 1 - x^{2} \right)\dx =$ 
\fillin{ $- \frac{1}{2}\left( 1 - x^{2} \right)^{2} + C$}.

\end{problem}           

\begin{problem}
如果$\displaystyle \frac{2}{1 + x^{2}}f(x) = \frac{\d}{{\dx}}\lbrack f(x)\rbrack^{2}$,
且$f\left( 0 \right) = 0$, 则 $f(x) =$ 
\fillin{ ${\arctan}x$}.

\end{problem}           

\begin{problem} $\displaystyle \int x^{2}\sqrt{1 + x^{3}}\dx =$ 
\fillin{ $\displaystyle \frac{2}{9}\left( 1 + x^{3} \right)^{\frac{3}{2}} + C$}.

\end{problem}           

\begin{problem}
若函数$\displaystyle f\left( x^{2} - 1 \right) = \ln\frac{x^{2}}{x^{2} - 2}$, 且$f\lbrack\varphi(x)\rbrack = \ln x$,
则 $\int \varphi(x)\dx =$ 
\fillin{ $x + 2\ln|x - 1| + C$}.

\end{problem}           

\begin{problem} 设$f'(\ln x) = 1 + x \ (x > 0)$, 则 $f(x) =$
\fillin{ $x + \e^{x} + C$}.

\end{problem}           

\begin{problem} $\displaystyle \int \frac{f(x) - xf'(x)}{f^{2}(x)}\dx =$ 
\fillin{ $\displaystyle \frac{x}{f(x)} + C$}.

\end{problem}           

\begin{problem}$f'(\cos x +2) = \sin^2 x + \tan^2 x$, 则$f\left( x \right) =$ 
\fillin{ $\displaystyle \frac{1}{2 - x} - \frac{1}{3}(x - 2)^{3} + C$}

\end{problem}           

\begin{problem} 设$f(x)$连续可导, 则 $\int f'(2x)\dx =$ 
\fillin{
$\displaystyle \left( \int f'(2x)\dx = \frac{1}{2}\int f'(2x)\d(2x) = \right)\frac{1}{2}f(2x) + C$}.

\end{problem}           

\begin{problem} $\displaystyle \int \frac{\dx}{\sqrt{a^{2} + x^{2}}} =$ \fillin{ $\ln\left| x + \sqrt{a^{2} + x^{2}} \right| + C$}
,其中$a$是正的常数.
\end{problem}           


\begin{problem} 己知 $\displaystyle \frac{\cos x}{x}$ 是 $f(x)$ 的一个原函数, 则
$\displaystyle \int f(x) \cdot \frac{\cos x}{x}\dx =$ 
\fillin{
$ \displaystyle \frac{1}{2}\left( \frac{\cos x}{x} \right)^{2} + C$}.

\end{problem}           

\begin{problem} 已知曲线上任一点的二阶导数$y' = 6x$, 且在曲线上 $(0,-2)$ 处的切线为
$2 x - 3 y = 6$, 则这条曲线方程为 
\fillin{ $3x^{3} + 2x - 3y - 6 = 0$}.
\end{problem}

\makepart{计算题}


\begin{problem}
	$\displaystyle \int \frac{\dx}{x^{2} - x - 6}$
	
	\begin{solution}
		$\text{原式}\displaystyle = \frac{1}{5}\ln\frac{x - 3}{x + 2} + C$
	\end{solution}
\end{problem}



 \begin{problem}
 	$\displaystyle\int \tan^{10} x \cdot \sec^2x\, \mathrm{d}x$

\begin{solution}
	$\text{原式}\displaystyle = \frac{1}{11}\tan{}^{11}x + C$
\end{solution}
 \end{problem}

\begin{problem}
	$\displaystyle \int \sin^5 x\,\mathrm{d}x$
	
\begin{solution}
	 $\text{原式}=\displaystyle -\cos x + \frac23 \cos^3 x -\frac15\cos^5x +C$
\end{solution}
\end{problem}

 \begin{problem}
 $\displaystyle \int \frac{\dx}{(\arcsin x)^{2}\sqrt{1 - x^{2}}}$

\begin{solution}
	$\text{原式}\displaystyle  = - \frac{1}{\arcsin x} + C$
\end{solution}
 \end{problem}

\begin{problem}
	$\displaystyle \int x \cdot \sqrt[4]{x + 9}\dx$

\begin{solution}
	$\text{原式}\displaystyle  = \frac{4}{9}\sqrt[4]{(x + 9)^{9}} - \frac{36}{5}\sqrt[4]{(x + 9)^{5}} + C$
\end{solution}
\end{problem}

\begin{problem}
	$\displaystyle \int \frac{{\dx}}{\sqrt{x^{2} + 2x + 2}}$

\begin{solution}
	$\text{原式}\displaystyle = \ln\left| x + 1 + \sqrt{x^{2} + 2x + 2} \right| + C$
\end{solution}
\end{problem}

\begin{problem}
	$\displaystyle \int \sqrt{x^{2} - a^{2}}{\dx}$


\begin{solution}
	$\text{原式}\displaystyle = \frac{1}{2}x\sqrt{x^{2} - a^{2}} - \frac{a^{2}}{2}\ln\left| x + \sqrt{x^{2} - a^{2}} \right| + C$
\end{solution}
\end{problem}

\begin{problem} $\displaystyle \int \frac{{\dx}}{\sqrt{1 + \e^{x}}}$

\begin{solution}
$\text{原式}\displaystyle = \ln\left( \frac{\sqrt{1 + \e^{x}} - 1}{\sqrt{1 + \e^{x}} + 1} \right) + C$
\end{solution}   
\end{problem}           


\begin{problem} 
	$\displaystyle \int \e^{\sqrt[3]{x}}{d}x$

\begin{solution}
	$\text{原式}\displaystyle = 3\e^{\sqrt[3]{x}}\left( \sqrt[3]{x^{2}} - 2 \cdot \sqrt[3]{x} + 2 \right) + C$
\end{solution}   
\end{problem}           


\begin{problem} 
	$\displaystyle \int \frac{x + 2}{x^{2} + 2x + 2}{d}x$

\begin{solution}
$\text{原式}\displaystyle = \frac{1}{2}\ln\left( x^{2} + 2x + 2 \right) + \arctan(x + 1) + C$

\end{solution}   
\end{problem}          


\begin{problem} 
	$\displaystyle {\int\left( x + \sqrt{x^{2} - 1} \right)}\,\mathrm{d}x$

\begin{solution} 
	$\text{原式}\displaystyle = x\ln\left( x + \sqrt{x^{2} - 1} \right) - \sqrt{x^{2} - 1} + C$

\end{solution}   
\end{problem}          


\begin{problem} 求$\displaystyle \int \frac{3^{x}5^{x}}{(25)^{x} - 9^{x}}{d}x$

\begin{solution}
	由条件得:
$$\begin{aligned}\text{原式}\displaystyle &= \int \frac{3^{x}5^{x}}{5^{2x} - 3^{2x}}\dx 
= \int \frac{\left( \frac{5}{3} \right)^{x}}{\left( \frac{5}{3} \right)^{2x} - 1}\dx \\
&= \frac{1}{\ln\frac{5}{3}}\int \frac{d\left( \frac{5}{3} \right)^{x}}{\left( \frac{5}{3} \right)^{2x} - 1} 
= \frac{1}{2\ln\frac{5}{3}}\ln\left| \frac{\left( \frac{5}{3} \right)^{x} - 1}{\left( \frac{5}{3} \right)^{x} + 1} \right| + C \\
&= \frac{1}{2(\ln 5 - \ln 3)}\ln\left| \frac{5^{x} - 3^{x}}{5^{x} + 3^{x}} \right| + C.
\end{aligned}
$$

\end{solution}   
\end{problem}           


\begin{problem} 求$\displaystyle \int \frac{\dx}{\left( 1 + \e^{x} \right)^{2}}$

\begin{solution} 
	由条件易知
	$$
	\text{原式}= \int \frac{\e^{x}\dx}{\e^{x}(1 + \e^{x})^{2}}$$
	
	令$\displaystyle \e^{x} = t$,则$\displaystyle \dx = \frac{1}{t}\dt$, 则
$$\begin{aligned}
\text{原式}&=\int \frac{\dt}{t(1 + t)^{2}} \\
 &= \int {\left( \frac{1}{t} - \frac{1}{(1 + t)^{2}} - \frac{1}{1 + t} \right)\dt} \\
 &= \ln|t| + \frac{1}{1 + t} - \ln|1 + t| + C \\
 &= x - \ln\left( 1 + \e^{x} \right) + \frac{1}{1 + \e^{x}} + C.
 \end{aligned}$$

\end{solution}   
\end{problem}           

\begin{problem} 求$\displaystyle \int \frac{x^{14}}{\left( x^{5} + 1 \right)^{4}}d{x.}$

\begin{solution}
	由条件易知:$\text{原式}\displaystyle = \frac{1}{5}\int \frac{x^{10}\dx^{5}}{\left( x^{5} + 1 \right)^{4}}$, 令$\displaystyle u = x^{5}$, 则

$$\begin{aligned}
\text{原式}  &= \frac{1}{5}\int \frac{u^{2}\du}{(1 + u)^{4}} \\
&= \frac{1}{5}\int {\frac{(u + 1)(u - 1) + 1}{(1 + u)^{4}}\du}\\
&=\displaystyle \frac15\int[\frac{u-1}{(1+u)^3}+\frac{1}{(1+u)^4}]\,\mathrm{d}u \\
&= \frac{1}{5}\int \left\lbrack \frac{1}{(1 + u)^{2}} - \frac{2}{(1 + u)^{3}} + \frac{1}{(1 + u)^{4}} \right\rbrack \du \\
&= \frac{1}{5}\left\lbrack - \frac{1}{1 + u} + \frac{1}{(1 + u)^{2}} - \frac{1}{3(1 + u)^{3}} \right\rbrack + C \\
&= \frac{1}{5}\left\lbrack - \frac{1}{1 + x^{5}} + \frac{1}{(1 + x^{5})^{2}} - \frac{1}{3(1 + x^{5})^{3}} \right\rbrack + C.
\end{aligned}$$

\end{solution}   
\end{problem}           

\begin{problem} 求$\displaystyle \int \frac{x^{2} \cdot \arccos x}{\sqrt{1 - x^{2}}}\dx$

\begin{solution} 令$ x = \cos t$,则$ \dx = - \sin t\dt$

$$\begin{aligned}
\text{原式} &= - \int t \cdot \frac{1 + \cos 2t}{2}\dt \\
&= - \frac{t^{2}}{4} - \frac{1}{2}\int t\cos 2t\dt \\
&= - \frac{t^{2}}{4} - \frac{1}{4}\int td(\sin 2t) \\
&= - \frac{t^{2}}{4} - \frac{1}{4}\left\lbrack t\sin 2t - \int\sin2t \dt \right\rbrack\\
&= - \frac{t^{2}}{4} - \frac{1}{4}t\sin 2t - \frac{1}{8}\cos 2t + C_{1} \\
&= - \frac{1}{4}(\arccos x)^{2} - \frac{1}{2}\arccos x \cdot x\sqrt{1 - x^{2}} - \frac{1}{8}\left( 2x^{2} - 1 \right) + C_{1} \\
&= - \frac{1}{4}(\arccos x)^{2} - \frac{1}{2}x\sqrt{1 - x^{2}}\arccos x - \frac{1}{4}x^{2} + C
\end{aligned}$$

\end{solution}   
\end{problem}           


\begin{problem} 计算积分$\displaystyle \int \frac{\sqrt{x^{2} + 2x + 2}}{(x + 1)^{2}}\dx$

\begin{solution} 原式$\displaystyle = \int \frac{\sqrt{(x + 1)^{2} + 1}}{(x + 1)^{2}}\dx$, 令$x + 1 = \tan t$, 则 $\displaystyle {\dx =}{\sec{}^{2}}t\dt$, 于是

$$\text{原式} = \ln\left| x + 1 + \sqrt{x^{2} + 2x + 2} \right| - \frac{\sqrt{x^{2} + 2x + 2}}{x + 1} + C.$$

\end{solution}   
\end{problem}           

\begin{problem} 计算积分
$\displaystyle \int \frac{\sqrt{x(x + 1)}}{\sqrt{x} + \sqrt{x + 1}}d{x.}$

\begin{solution} 分母有理化,则
$$
\begin{aligned}
\text{原式} & = \int \frac{\sqrt{x(x{\ +\ }1)}(\sqrt{x} - \sqrt{x + 1})}{x - (x + 1)}\dx\\
& = - \int \left\lbrack x\sqrt{x + 1} - \sqrt{x}(x + 1) \right\rbrack \dx\\
& = - \int {(x + 1 - 1)}\sqrt{x +1}{\ d}(x + 1){\ +}\int {(\sqrt{x} + x\sqrt{x})}\dx \\
& = - \frac{2}{5}(x + 1)^{5/2} + \frac{2}{3}(x + 1)^{3/2} + \frac{2}{3}x^{3/2} + \frac{2}{5}x^{5/2} + C
 \end{aligned}
 $$

\end{solution}   
\end{problem}           

\begin{problem} 求不定积分$\displaystyle \int \frac{x\dx}{(x + 2)\sqrt{x^{2} + 4x - 12}}$.

\begin{solution} 
易知	$\sqrt{x^{2} + 4x - 12} = \sqrt{(x + 2)^{2} - 4^{2}}$, 令$\displaystyle x + 2 = 4\sec t,$则$\displaystyle \dx = 4\sec t \cdot \tan t\dt,$ 于是我们有

$$\begin{aligned} \text{原式} &= \int \frac{(4\sec t - 2)}{4\sec t \cdot 4\tan t} \cdot 4\sec t \cdot \tan t\dt \\
&= \int \sec t\dt -\frac12\int \dt\\
&= \ln|\sec t + \tan t| - \frac{1}{2}t + C_{1} \\
&= \ln\left| \frac{x + 2}{4} + \frac{\sqrt{x^{2} + 4x - 12}}{4} \right| - \frac{1}{2}\arccos\frac{4}{x + 2} + C_{1}\\
&= \ln\left| x + 2 + \sqrt{x^{2} + 4x - 12} \right| - \frac{1}{2}\arccos\frac{4}{x + 2} + C.
\end{aligned}
$$

\end{solution}   
\end{problem}           


\begin{problem} 求不定积分$\displaystyle \int \frac{\dx}{a\sin x + b\cos x}$.

\begin{solution} 令$ a = A\cos\varphi,b = A\sin\varphi,$ 其中
$\displaystyle A = \sqrt{a^{2} + b^{2}},$ 则

$$ \tan\frac{\varphi}{2} = \frac{1 - \cos\varphi}{\sin\varphi} = \frac{\sqrt{a^{2} + b^{2}} - a}{b},$$

于是

$$\begin{aligned}\text{原式} &= \frac{1}{\sqrt{a^{2} + b^{2}}}\int \frac{\dx}{\sin\varphi\cos x + \cos\varphi\sin x}\\
 &= \frac{1}{\sqrt{a^{2} + b^{2}}}\int \frac{\dx}{\sin(x + \varphi)} \\
 &= \frac{1}{\sqrt{a^{2} + b^{2}}}\ln\left| \frac{\tan\frac{x}{2} + \tan\frac{\varphi}{2}}{1 - \tan\frac{x}{2}\tan\frac{\varphi}{2}} \right| + C \\
 & = \frac{1}{\sqrt{a^{2} + b^{2}}}\ln\left| \frac{b\tan\frac{x}{2} - a + \sqrt{a^{2} + b^{2}}}{b - \tan\frac{x}{2}(\sqrt{a^{2} + b^{2}} - a)} \right| + C.
 \end{aligned}$$

\end{solution}   \end{problem}           

\begin{problem} 求不定积分$\displaystyle \int\dfrac{\dx}{\sin^3 x\cos x}$

\begin{solution} 

$\text{原式}\displaystyle {= \ln|\csc 2x - \cot 2x| - \frac{1}{2}}{\sin{}^{- 2}}x + C$
$\displaystyle {= \ln|\tan x| - \frac{1}{2}}{\csc{}^{2}}x + C$

\end{solution}   

\end{problem}           


\begin{problem} 求不定积分 $\displaystyle \int \frac{\sqrt{x + 1} - 1}{\sqrt{x + 1} + 1}\dx$

\begin{solution}
$$\begin{aligned} \int \frac{\sqrt{x + 1} - 1}{\sqrt{x + 1} + 1}\dx &= \int \frac{x + 1 - 2\sqrt{x + 1} + 1}{(x + 1) - 1}\dx \\
&= \int {\left( 1 + \frac{2}{x} - 2\frac{\sqrt{x + 1}}{x} \right)\dx} \\
&= x + 2\ln\left| x \right| - 2\int {\frac{\sqrt{x + 1}}{x}\dx}.
\end{aligned}$$

令$\displaystyle \sqrt{x + 1} = u$,则$\displaystyle x = u^{2} - 1$,$\displaystyle \dx = 2u\du$,于是

$$\begin{aligned}
 \int \frac{\sqrt{x + 1}}{x}\dx &=\int {\frac{u}{u^{2} - 1}2u\du}{=2}\int \frac{u^{2} - 1 + 1}{u^{2} - 1}\du\\
& = 2u + \int \frac{\du}{u^{2} - 1} = 2u + \ln\left| \frac{u - 1}{u + 1} \right| + C\\
& = 2\sqrt{x + 1} + \ln\left| \frac{\sqrt{x + 1} - 1}{\sqrt{x + 1} + 1} \right| + C
\end{aligned}$$

故
$$\begin{aligned}\text{原式} &= x + 2\ln|x| - 4\sqrt{x + 1} - 2\ln\left| \frac{\sqrt{x + 1} - 1}{\sqrt{x + 1} + 1} \right| + C \\ &= x - 4\sqrt{x + 1} + 4\ln(\sqrt{x + 1} + 1) + C.\end{aligned}$$

\end{solution}   
\end{problem}          


 \begin{problem} 求不定积分 $\displaystyle \int \frac{\dx}{1 + \tan x}$.

\begin{solution} $\text{原式}\displaystyle \int \frac{\dx}{1 + \tan x}$

\end{solution}   
\end{problem}           


\begin{problem} 求$\displaystyle \int \frac{x^{2} - 1}{\sqrt{2x - 1}}\dx$

\begin{solution} 令$\sqrt{2x - 1} = u,2x - 1 = u^{2},x = \frac{1 + u^{2}}{2}$,
则$ \dx = u\du$, 于是

$$\begin{aligned}
 \text{原式} & = \int \frac{\left( \frac{1 + u^{2}}{2} \right)^{2} - 1}{u} \cdot u\du \\
 &= \int \left\lbrack \frac{\left( 1 + u^{2} \right)^{2}}{4} - 1 \right\rbrack \du\\
 & = \frac{1}{4}\int \left( 1 + 2u^{2} + u^{4} - 4 \right)\du = \frac{1}{4}\left( - 3u + \frac{2}{3}u^{3} + \frac{u^{5}}{5} \right) + C\\
 & = - \frac{3}{4}\sqrt{2x - 1} + \frac{1}{6}\sqrt{(2x - 1)^{3}} + \frac{1}{20}\sqrt{(2x - 1)^{5}} + C
  \end{aligned}$$
\end{solution}   
\end{problem}


\makepart{综合与应用题}

\begin{problem}
一质点作直线运动,已知其加速度为$a = 12t^{2} - 3\sin t.$ 如果 $v\left( 0 \right) = 5$,$s\left( 0 \right) = - 3$,求:
	
	(1) 速度$v$与时间$t$的关系;
	
	(2)位移$s$与时间$t$的关系.
	
	\begin{solution}
		(1) 由条件知
		$$v = \int a\dt = \int \left( 12t^{2} - 3\sin t \right)\dt = 4t^{3} + 3\cos t + C_{1}.$$
	
	又$v\left( 0 \right) = 5$,得$5 = 3 + C_{1}$,$C_{1} = 2$,故$v = 4t^{3} + 3\cos t + 2.$
	
	(2) 由条件知
	$$s = \int {v dt = \int \left( 4t^{3} + 3\cos t + 2 \right)}dt = t^{4} + 3\sin t + 2t + C_{2}.$$
	
	又$s\left( 0 \right) = - 3$,得$C_{2} = - 3$,故$s = t^{4} + 3\sin t + 2t - 3.$
	\end{solution}
\end{problem}

\begin{problem}
	一曲线通过点$\left( \e^{2},3 \right)$,且在任一点处的切线的斜率等于该点横坐标的倒数,求该曲线的方程.
	
\begin{solution}
	由条件知
	 $$ \frac{{\dy}}{{\dx}} = \frac{1}{x},\ y = \int {\frac{1}{x}\dx = \ln\left| x \right| + C},$$
	
	又曲线过点$\left( \e^{2},3 \right)$,于是$3 = \ln \e^{2} + C$,$C = 1$,故所求方程为
	
	$$y = \ln\left| x \right| + 1.$$
\end{solution}
\end{problem}

\begin{problem}
	导出计算积分$I_{n} = \int {\tan}^{n}{x\d x}$的递推公式,其中$n$为自然数.
	
	\begin{solution}
		由条件
		$$\begin{aligned} I_{n} &=\int \tan ^{n} x \d x=\int \tan ^{n-2} x\left(\sec ^{2} x-1\right) \d x \\ &=\int \tan ^{n-2} x \sec ^{2} x \d x-\int \tan ^{n-2} x \d x \\ &=\int \tan ^{n-2} x \d \tan x-\int \tan ^{n-2} x \d x \\ &=\frac{\tan ^{n-1} x}{n-1}-I_{n-2}(n \geq 2)  \end{aligned}$$
	\end{solution}
$$I_{1} =\int \tan x \d x=-\ln |\cos x|+C, \ I_{0}=\int \d x=x+C$$
\end{problem}

\begin{problem}
	若$f\left( x \right)$的原函数为$\dfrac{\ln x}{x}$,问$f\left( x \right)$与$\dfrac{\ln x}{x}$间有什么关系?并求$\int{xf'\left( x \right)}{\d}x$.
	
	\begin{solution}
		由条件知
		$$f\left( x \right) = \left( \frac{\ln x}{x} \right)' = \frac{1 - \ln x}{x^{2}},$$
		故
		$$\int{f\left( x \right){\dx}} = \frac{\ln x}{x} + C.$$
	于是
	$$\int{xf'\left( x \right){\dx}} = \int x\d f\left( x \right) ={xf}\left( x \right) - \int{f\left( x \right){\dx}} = \frac{1 - 2\ln x}{x} + C.$$
	\end{solution}
\end{problem}


\begin{problem}
	设$y = y\left( x \right)$是由方程$y^{2}\left( x - y \right) = x^{2}$所确定的隐函数,试求$\displaystyle \int\frac{{dx}}{y^{2}}.$
	
	\begin{solution}
		设$y = t \cdot x$,代入方程得$t^{2}x\left( 1 - t \right) = 1$,即$x = \dfrac{1}{t^{2}\left( 1 - t \right)}$,则
	
	$$\dx = \frac{3t - 2}{t^{3}(1 - t)^{2}}{\dt},\ y = \frac{1}{t\left( 1 - t \right)},$$
	于是
	$$\int\frac{{dx}}{y^{2}} = \int {t^{2}\left( 1 - t \right)^{2} \cdot \frac{3t - 2}{t^{3}\left( 1 - t \right)^{2}}}dt = \int {\left( 3 - \frac{2}{t} \right)dt = 3t - 2\ln\left| t \right|} + C = \frac{3y}{x} - 2\ln\left| \frac{y}{x} \right| + C.$$
	\end{solution}
\end{problem}


\begin{problem}
	设$\displaystyle f\left( \sin^{2}x \right) = \frac{x}{\sin x}$,求$\displaystyle \int{\frac{\sqrt{x}}{\sqrt{1 - x}}f\left( x \right)}{\dx}$.
	
	\begin{solution}
		设$\sin^{2}x = t$,即$\sin x = \sqrt{t}$,$x = \arcsin\sqrt{t}$,$f\left( t \right) = \dfrac{\arcsin\sqrt{t}}{\sqrt{t}}$,则
	
	$$\begin{aligned}
	\int{\frac{\sqrt{x}}{\sqrt{1 - x}}f\left( x \right)}\dx 
	& = \int \frac{\sqrt{x}}{\sqrt{1 - x}} \cdot \frac{\arcsin\sqrt{x}}{\sqrt{x}}\dx\\
	& = - 2\int {\arcsin\sqrt{x}}\d\sqrt{1 - x}\\
	& = - 2\sqrt{1 - x}\arcsin\sqrt{x} + 2\int {\sqrt{1 - x}\frac{\frac{1}{2\sqrt{x}}}{\sqrt{1 - x}}}{\dx}\\
	& = - 2\sqrt{1 - x}\arcsin\sqrt{x} + 2\sqrt{x} + C.\end{aligned}$$
	\end{solution}
\end{problem}

\begin{problem}
	设$f\left( \ln x \right) = \dfrac{\ln\left( 1 + x \right)}{x}$,计算$\int{f\left( x \right)}{\dx}$.
	
		\begin{solution}
	设$\ln x = t$,则$x = \e^{t}$,$f\left( t \right) = \dfrac{\ln\left( 1 + \e^{t} \right)}{\e^{t}}$, 于是
		$$\begin{aligned} 
		\int f(x) \d x &=\int \frac{\ln \left(1+\e^{x}\right)}{\e^{x}} \d x\\
		&=-\int \ln \left(1+e^{x}\right) \d \e^{-x} \\ 
		&=-\e^{-x} \ln \left(1+e^{x}\right)+\int \frac{1}{1+\e^{x}} \d x \\ 
		&=-\e^{-x} \ln \left(1+\e^{x}\right)+\int\left(1-\frac{\e^{x}}{1+\e^{x}}\right) \d x \\ &=x-\left(1+\e^{-x}\right) \ln \left(1+\e^{x}\right)+C 
		\end{aligned}$$
	\end{solution}
\end{problem}

\begin{problem}
	设$f\left( x^{2} - 1 \right) = \ln\dfrac{x^{2}}{x^{2} - 2}$,且$f\left\lbrack \varphi\left( x \right) \right\rbrack = \ln x$,求$\int {\varphi\left( x \right)}{\dx}$.
	
	\begin{solution}
		因为
		$$ f\left( x^{2} - 1 \right) = \ln\dfrac{\left( x^{2} - 1 \right) + 1}{\left( x^{2} - 1 \right) - 1},$$从而有$$f\left( x \right) = \ln\dfrac{x + 1}{x - 1}.$$ 又
	$$f\left\lbrack \varphi\left( x \right) \right\rbrack = \ln\dfrac{\varphi\left( x \right) + 1}{\varphi\left( x \right) - 1} = \ln x。$$
	于是
	$$\frac{\varphi\left( x \right) + 1}{\varphi\left( x \right) - 1} = x \Rightarrow \varphi\left( x \right) = \frac{x + 1}{x - 1}.$$
	故
	$$\int\varphi(x)\,\mathrm{d}x = x + 2\ln\left| x - 1 \right| + C.$$
	\end{solution}
\end{problem}



\begin{problem} 设$f\left( x \right) = \left\{ \begin{matrix}
x^{2}, x \leq 0 \\
\sin x,x > 0 \\
\end{matrix} \right.\ $,求$f\left( x \right)$的不定积分.

\begin{solution} 
	由条件易得
	$$\int{f\left( x \right){\dx}} = \left\{ \begin{matrix}
\dfrac{x^{3}}{3} + C_{1},& x \leq 0; \\
-\cos x + C_{2}, &x > 0. \\
\end{matrix} \right. $$

由原函数的连续性得,$\displaystyle \lim_{x \rightarrow 0^{-}}\left( \frac{x^{3}}{3} + C_{1} \right) = \lim_{x \rightarrow 0^{+}}\left( - \cos x + C_{2} \right)$
, 从而
$C_{1} = C_{2} - 1$,再令$C_{1} = C_{2} - 1 = C$即得

$$\int{f\left( x \right)\d x} = \left\{ \begin{matrix}
\dfrac{x^{3}}{3} + C, &x \leq 0, \\
1 - \cos x + C, &x > 0. \\
\end{matrix} \right.$$

\end{solution}   
\end{problem}

\begin{problem}
在什么条件下,积分$\int\dfrac{ax^{2} + bx + c}{x^{3}\left( x - 1 \right)^{2}}{\dx}$表示有理函数?

\begin{solution}
由$$\frac{ax^{2} + bx + c}{x^{3}\left( x - 1 \right)^{2}} = \frac{A_{1}}{x} + \frac{A_{2}}{x^{2}} + \frac{A_{3}}{x^{3}} + \frac{B_{1}}{x - 1} + \frac{B_{2}}{\left( x - 1 \right)^{2}},$$

可知,当$A_{1} \neq 0$,$B_{1} \neq 0$时,$\dfrac{A_{1}}{x}$, $\dfrac{B_{1}}{x - 1}$的积分为对数函数,因此要使该积分为有理函数,必须$A_{1} = B_{1} = 0$, 故
$$\frac{ax^{2} + bx + c}{x^{3}\left( x - 1 \right)^{2}} = \frac{A_{2}}{x^{2}} + \frac{A_{3}}{x^{3}} + \frac{B_{2}}{\left( x - 1 \right)^{2}},$$
于是
$$ax^{2} + bx + c \equiv A_{2}x\left( x - 1 \right)^{2} + A_{3}\left( x - 1 \right)^{2} + B_{2}x^{3}.$$

令$x = 0$,得$A_{3} = c\ \text{\ding{172}}$ ;令$x = 1$,得$B_{2} = a + b + c\ \text{\ding{173}}$;并结合比较令$x^{3}$,$x^{2}$的系数,得

$x^{3}:A_{2} + B_{2} = 0\ \text{\ding{174}}$ ③;$x^{2}:A_{3} - 2A_{2} = a\ \text{\ding{175}}$ ;

由\ding{172}--\ding{175},可得所求条件为$a + 2b + 3c = 0.$

\end{solution}   \end{problem}\begin{problem}
设$f\left( x \right)$是单调连续函数,$f^{- 1}\left( x \right)$是它的反函数,且
$\int{f\left( x \right)}\dx = F\left( x \right) + C.$ 求$\int {f^{- 1}\left( x \right)}{\dx}.$

\begin{solution} 因为 $\quad x = f\left( f^{- 1}\left( x \right) \right)$,所以
$$\begin{aligned} \int f^{-1}(x) d x &=x f^{-1}(x)-\int x d f^{-1}(x) \\ &=x f^{-1}(x)-\int f\left(f^{-1}(x)\right) d f^{-1}(x) \\ &=x f^{-1}(x)-F\left(f^{-1}(x)\right)+\mathrm{C} \end{aligned}$$

\end{solution}   \end{problem}

\begin{problem}
设$\displaystyle f'\left( x\tan\frac{x}{2} \right) = \left( x + \sin x \right)\tan\frac{x}{2} + \cos x$,求$f\left( x \right)$.

\begin{solution}
	由条件
$$f'\left( x\tan\frac{x}{2} \right) = x\tan\frac{x}{2} + \sin x\tan\frac{x}{2} + \cos x = x\tan\frac{x}{2} + 1.$$
令$\displaystyle u = x\tan \frac{x}{2}$, 则$f'\left( u \right) = u + 1$,故
$$f\left( u \right) = \int \left( u + 1 \right)du = \frac{u^{2}}{2} + u + C,$$
所以
$$ f\left( x \right) = \frac{x^{2}}{2} + x + C.$$
\end{solution}   \end{problem}



\makepart{分析与证明题}

\begin{problem}
	设 \(F\left( x \right)\) 是 \(f(x)\) 的一个原函数,
\(f\left( x \right)\) 可微且其反函数 \(f^{- 1}\left( x \right)\) 存在,
则$$\int f^{- 1}\left( x \right)\dx = xf^{- 1}\left( x \right) - F\left\lbrack f^{- 1}\left( x \right) \right\rbrack + C.$$

\begin{solution}
由分部积分公式得

$$\int f^{- 1}\left( x \right)dx = xf^{- 1}\left( x \right) - \int xd\left\lbrack f^{- 1}\left( x \right) \right\rbrack,$$

设 \(t = f^{- 1}\left( x \right)\), 则 \(x = f\left( t \right)\),于是
$$\int xd\left\lbrack f^{- 1}\left( x \right) \right\rbrack = \int f\left( t \right)dt= F\left( t \right) + C = F\left\lbrack f^{- 1}\left( x \right) \right\rbrack + C,$$
所以
$$\int f^{- 1}(x)dx = xf^{- 1}(x) - F\left\lbrack f^{- 1}(x) \right\rbrack + C.$$
\end{solution}
\end{problem}

\begin{problem}
	证明函数 \(\dfrac{1}{2}\e^{2x}\), \(\e^{x}{\mathrm{sh}\ }x\) 和 \(\e^{x}{\mathrm{ch}}x\)
都是 \(\dfrac{\e^{x}}{\mathrm{ch}x - \mathrm{sh}x}\) 的原函数.

\begin{solution}
	易知
	$$ \left( \dfrac{1}{2}\e^{2x} \right)' = \e^{2x} = \dfrac{\e^{x}}{\dfrac{\e^{x} + \e^{- x}}{2} - \dfrac{\e^{x} - \e^{- x}}{2}} = \frac{\e^{x}}{\mathrm{ch}x - \mathrm{sh}x}$$
所以
\(\dfrac{1}{2}\e^{2x}\) 是 \(\dfrac{\e^{x}}{\mathrm{ch}x - \mathrm{sh}x}\)
的原函数. 又因为
$$\left( \e^{x}\mathrm{sh}x \right)' = \e^{x}\mathrm{sh}x + \e^{x}\mathrm{ch}x = \e^{x}(\mathrm{ch}x + \mathrm{sh}x)= \frac{\e^{x} \cdot \left( {\mathrm{ch}}^{2}x - {\mathrm{sh}}^{2}x \right)}{\mathrm{ch}x - \mathrm{sh}x} = \frac{\e^{x}}{\mathrm{ch}x - \mathrm{sh}x}$$
所以
\(\e^{x}\mathrm{sh}x\) 是 \(\dfrac{\e^{x}}{\mathrm{ch}x - \mathrm{sh}x}\) 的原函数.

同理, 易证 \(\e^{x}\mathrm{ch}x\) 也是 \(\dfrac{\e^{x}}{\mathrm{ch}x - \mathrm{sh}x}\) 的原函数.
\end{solution}
\end{problem}

\begin{problem}
	设 \(f(x) = \mathrm{sgn} x = \left\{ \begin{matrix}
1 & x > 0 \\
0 & x = 0 \\-1 & x < 0 \\
\end{matrix} \right.\ \) ,证明:\(y = \dfrac{x^{2}}{2}\mathrm{sgn}x\) 是
\(y = |x|\) 的原函数.

\begin{solution}
	易知
	$$y = \dfrac{x^{2}}{2}\mathrm{sgn}x = \left\{ \begin{matrix}
0.5x^{2} & x > 0 \\
0 & x = 0 \\-0.5x^{2} & x < 0 \\
\end{matrix} \right. ,\ \ y' = \left\{ \begin{matrix}
x & x > 0 \\
& \\-x & x < 0 \\
\end{matrix} \right.$$
于是
$$\left. \ y' \right|_{x = 0} = \lim_{x \rightarrow 0}\mspace{2mu}\frac{0.5x^{2} \cdot \mathrm{sgn}x - 0}{x - 0} = \lim_{x \rightarrow 0}\mspace{2mu} 0.5x \cdot \mathrm{sgn}x = 0$$

即: \(y' = |x|\),所以 \(y = \dfrac{x^{2}}{2}\mathrm{sgn}x\) 是 \(y = |x|\)
的原函数.
\end{solution}
\end{problem}