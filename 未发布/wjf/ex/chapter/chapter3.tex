\chapter{导数、微分、边际与弹性}


\makepart{单项选择题}


\begin{problem}设 $f\left( x \right)$ 在 $x = 1$ 处可导,且
$f'\left( 1 \right) = 2$,则
$\displaystyle \lim_{x \rightarrow 0}{\frac{f\left( 1 + x \right) - f(1 - x)}{x} =}$
\pickin{C}.

\begin{abcd} \item 1

\item 2

\item 4

\item 3

\end{abcd}

\kpoint{ 导数概念}
\end{problem}           


\begin{problem}函数 $f\left( x \right) = \left| x \right|^{3}$ 在 $x = 0$
处满足下列哪个结论 \pickin{D}.

\begin{abcd} \item 极限不存在

\item 极限存在,不连续

\item 连续,不可导

\item 可导

\end{abcd}

\kpoint{ 导数概念}

\end{problem}           


\begin{problem}函数 $f\left( x \right)$ 在区间 $(a,b)$ 内连续是 $f(x)$ 在
$(a,b)$ 内可导的\pickin{B}.

\begin{abcd} \item 充分但非必要条件

\item 必要但非充分条件

\item 充分必要条件

\item 既非充分又非必要条件

\end{abcd}

\kpoint{ 导数连续关系}
\end{problem}           


\begin{problem}设函数 $f(x)$ 可导,记
$g\left( x \right) = f\left( x \right) + f( - x)$,则导数
$g'\left( x \right)$ 为\pickin{A}.

\begin{abcd} 
\item 奇函数

\item 偶函数

\item 非奇非偶

\item 奇偶性不定

\end{abcd} 

\kpoint{ 导函数}
\end{problem}           

\begin{problem}函数 $f(x) = \left\{ \begin{matrix}
\displaystyle \frac{x}{1 - \e^{\frac{1}{x}}} & x \neq 0 \\
0 & x = 0 \\
\end{matrix} \right.\ $, 在 $x = 0$ 处\pickin{B}.

\begin{abcd} 
	
\item 不连续

\item 连续但不可导

\item 可导,且 $f'\left( 0 \right) = 0$

\item 可导,且 $f'\left( 0 \right) = 1$

\end{abcd}

\kpoint{ 单侧导数}
\end{problem}           


\begin{problem}设 $\e^{2x}$ 为 $f(x)$ 的导函数,则 $f''\left( x \right) =$ \pickin{B}.

\begin{abcd} 
	
\item $\e^{2x}$

\item $\ {2\e}^{2x}$

\item ${4\e}^{2x}$

\item 0

\end{abcd}

\kpoint{ 高阶导数}
\end{problem}           


\begin{problem}设 $f'\left( 0 \right) = 2$,则当 $x \rightarrow 0$
时,$f\left( x \right) - f(0)$ 是 $x$ 的\pickin{B}.

\begin{abcd} \item 低阶无穷小量

\item 同阶无穷小量

\item 高阶无穷小量

\item 等价无穷小量

\end{abcd}

\kpoint{ 导数概念}
\end{problem}           

\begin{problem}设 $f\left( x \right) = x\ln\ 2x$ 在 $x_{0}$ 处可导,且
$f'\left( x_{0} \right) = 2$,则 $f\left( x_{0} \right) =$\pickin{B}.

\begin{abcd} \item $1$

\item $\displaystyle \frac{\e}{2}$

\item $\displaystyle \frac{2}{\e}$

\item $\e^{2}$

\end{abcd}

\kpoint{ 导函数}
\end{problem}           

\begin{problem}曲线 $y = x\ln x - x$ 在 $x = \e$ 处的切线方程是\pickin{B}.

\begin{abcd} \item $y = \e - x$

\item $\ y = x - \e$

\item $\ y = x - \e + 1$

\item $y = \e + x$

\end{abcd}

\kpoint{ 导数概念}
\end{problem}           


\begin{problem}设 $f(x)$ 可导且 $f'\left( - 2 \right) = 2$,又
$y = f( - x^{2})$,则 ${\d y|}_{x = \sqrt{2}} =$ \pickin{D}.

\begin{abcd} 
	
\item $2\dx$

\item $- 2\dx$

\item $4\sqrt{2}\dx$

\item $- \ 4\sqrt{2}\dx$

\end{abcd}

\kpoint{ 一阶微分形式不变性}
\end{problem}           


\begin{problem}设 $f\left( 0 \right) = 0$,且 $f'(0)$ 存在,则
$\displaystyle \lim_{x \rightarrow 0}{\frac{f\left( x \right)}{x} =}$ \pickin{B}.

\begin{abcd} \item $f'\left( x \right)$

\item $f'\left( 0 \right)$

\item $f\left( 0 \right)$

\item $\frac{1}{2}f\left( 0 \right)$

\end{abcd}

\kpoint{ 导数概念}
\end{problem}           


\begin{problem}设 $f(x) = \left\{ \begin{matrix}
\displaystyle x^{2}\sin\frac{1}{x} & x \neq 0 \\
0 & x = 0 \\
\end{matrix} \right.\ $ ,则该函数在 $x = 0$ 处\pickin{D}.

\begin{abcd} 
\item 极限不存在

\item 极限存在但不连续

\item 连续但不可导

\item 可导

\end{abcd}

\kpoint{ 导数连续关系}
\end{problem}           

\begin{problem}设 $y = f(x)$,已知
$\displaystyle \lim_{x \rightarrow 0}\mspace{2mu}\frac{f\left( x_{0} \right) - f\left( x_{0} + 2x \right)}{6x} = 3$,则
${\dy|}_{x = x_{0}} =$\pickin{A}.

\begin{abcd} 
\item $- 9\dx$

\item $18\dx$

\item $- 3\dx$

\item $2\dx$

\end{abcd}

\kpoint{ 可导与微分的关系}
\end{problem}           


\begin{problem}设 $\displaystyle y = x(x - 1)(x - 2)(x - 3)(x - 4)(x - 5)$, $\displaystyle {y'|}_{x = 0}=$
\pickin{B}.

\begin{abcd} 
	
\item 0

\item $- 5!$

\item -5

\item -15

\end{abcd}

\kpoint{ 导数的四则运算}
\end{problem}           

\begin{problem}设可微函数 $y = f(x)$,如果 $f'\left( x_{0} \right) = 0.5$,则当
$\Delta x \rightarrow 0\ $ 时,该函数在 $x = x_{0}$ 处的微分
${\dy}$ 是\pickin{B}.

\begin{abcd} \item $\Delta x$ 的等价无穷小

\item $\Delta x$ 的同阶但不等价的无穷小

\item $\Delta x$ 的低阶无穷小

\item $\Delta x$ 的高阶无穷小

\end{abcd}

\anlys{${\dy|}_{x = x_{0}} = 0.5\Delta x$}

\kpoint{ 可导与微分的关系}
\end{problem}           


\begin{problem}下列函数中,在点 $x = 0$ 处可导的是\pickin{B}.

\begin{abcd} \item $f\left( x \right) = |x|$

\item $f\left( x \right) = |x - 1|$

\item $f\left( x \right) = |\sin x|$

\item
$\ f(x) = \left\{ \begin{matrix} x^{2} & x \leq \ 0 \\ x & x > 0 \\ \end{matrix} \right.\ $

\end{abcd}

\kpoint{ 单侧导数}
\end{problem}           

\begin{problem}设周期函数 $f\left( x \right)$ 在 $( - \infty, + \infty)$
内可导,周期为4,又
$\displaystyle \lim_{x \rightarrow 0}\mspace{2mu}\frac{f(1) - f(1 - x)}{2x} = - 1$,则
$y = f\left( x \right)$ 在点 $(5,f(5))$ 处的切线的斜率为\pickin{D}.

\begin{abcd} 
	\item $\displaystyle \frac{1}{2}$ \item $0$ \item $- 1$ \item $- 2$

\end{abcd}

\anlys{易知
$$\lim_{\Delta x \rightarrow 0}\mspace{2mu}\frac{f(5 + \Delta x) - f(5)}{\Delta x} = \lim_{\Delta x \rightarrow 0}\mspace{2mu}\frac{f(1 + \Delta x) - f(1)}{\Delta x} $$
令$x=-x$,则有
$$\lim_{x \rightarrow 0}\mspace{2mu}\frac{f(1 - x) - f(1)}{- x} = \lim_{x \rightarrow 0}\mspace{2mu}\frac{f(1) - f(1 - x)}{2x} \cdot 2 = - 2.$$}

\kpoint{ 导数概念}
\end{problem}           


\begin{problem}设 $f(x) = \left\{ \begin{matrix}
\displaystyle \frac{1 - \cos x}{\sqrt{x}}\ \ \ \  x > 0 \\
 x^{2}g(x)\ \ \ \  x \leq 0 \\
\end{matrix} \right.\ $ 其中 $g'(x)$ 是有界函数,则 $f(x)$ 在
$x = 0$ 处\pickin{D}.

\begin{abcd} 
	
\item 极限不存在

\item 极限存在,但不连续

\item 连续,但不可导

\item 可导

\end{abcd}

\anlys{$g'\left( x \right)$ 是有界函数,$g(x)$ 连续
$\Rightarrow {g}\left( x \right)$ 有界.

\begin{equation*}
\begin{split}
\lim_{x \rightarrow 0^{-}}\mspace{2mu}\frac{f(x) - f(0)}{x - 0} &= \lim_{x \rightarrow 0^{-}}\mspace{2mu}\frac{x^{2} \cdot g(x) - 0^{2} \cdot g(0)}{x} = 0, \\
\lim_{x \rightarrow 0^{+}}\mspace{2mu}\frac{f(x) - f(0)}{x - 0} &= \lim_{x \rightarrow 0^{+}}\mspace{2mu}\frac{(1 - \cos x)/\sqrt{x} - 0^{2} \cdot g(0)}{x} \\
& = \lim_{x \rightarrow 0^{+}}\mspace{2mu}\frac{1 - \cos x}{x^{3/2}} = \lim_{x \rightarrow 0^{+}}\mspace{2mu}\frac{1}{2}x^{\frac{1}{2}} = 0,
\end{split}
\end{equation*}
得$f'(0)$存在.}

\kpoint{ 单侧导数}

\end{problem}          


 \begin{problem}设函数 $f(x) = \left\{ \begin{matrix}
\sqrt{|x|}\sin\left( 1/x^{2} \right) & x \neq 0 \\
0 & x = 0 \\
\end{matrix} \right.\ $,则 $f(x)$ 在 $x = 0$ 处\pickin{C}.

\begin{abcd} \item 极限不存在

\item 极限存在但不连续

\item 连续但不可导

\item 可导

\end{abcd}

\anlys{易知

$$\lim_{x \rightarrow 0}\mspace{2mu} f(x) = \lim_{x \rightarrow 0}\mspace{2mu}\sqrt{\left| x \right|}\sin\left( \frac{1}{x^{2}} \right) = 0,\ \ \ \left| \sin\left( 1/x^{2} \right) \right| \leq 1 \Rightarrow
\lim_{x \rightarrow 0}\mspace{2mu} f(x) = 0 = f(0)$$
故$f(x)$在$x=0$处连续;
又
\begin{equation*}
\lim_{x \rightarrow 0^{+}}\mspace{2mu}\frac{f(x) - f(0)}{x - 0} = \lim_{x \rightarrow 0^{+}}\mspace{2mu}\frac{\sqrt{|x|}\sin\left( 1/x^{2} \right)}{x} = \lim_{x \rightarrow 0^{+}}\mspace{2mu}\frac{1}{\sqrt{x}}\sin\frac{1}{x^{2}}
\end{equation*}
极限不存在,故在$x=0$处不可导.}

\kpoint{ 单侧导数}
\end{problem}           

\begin{problem}设
$F\left( x \right) = \max\left[ f_{1}\left( x \right),f_{2}(x)\right]$,
$0 < x < 2$,其中 $f_{1}(x) = x$, $f_{2}(x) = x^{2}$,则\pickin{D}.

\begin{abcd} \item $F'(x) = \left\{ \begin{matrix}
 1\ \ \ \  0 < x < 0.5 \\
 2x\ \ \ \  0.5 < x < 2 \\
\end{matrix} \right.\ $

\item $F'(x) = \left\{ \begin{matrix}
 1\ \ \ \  0 < x \leq 1 \\
 2x\ \ \ \  1 < x < 2 \\
\end{matrix} \right.\ $

\item $\ F'(x) = \left\{ \begin{matrix}
 1\ \ \ \  0 < x < 1 \\
 2x\ \ \ \  1 \leq x < 2 \\
\end{matrix} \right.\ $

\item $\ F'(x) = \left\{ \begin{matrix}
 1\ \ \  0 < x < 1 \\
 2x\ \ \ \  1 < x < 2 \\
\end{matrix} \right.\ $

\end{abcd}

\anlys{$F(x) = \left\{ \begin{matrix}
 x\ \ \ \  0 < x \leq 1 \\
 x^{2}\ \ \ \  1 < x < 2 \\
\end{matrix} \right.\ $,
$F_{+}'\left( 1 \right) = 2F_{-}'\left( 1 \right) = 2$, 故$F'(1)$不存在, 所以
$$F'(x) = \left\{ \begin{matrix}
1\ \ \ \  0 < x < 1 \\
2x\ \ \ \  1 < x < 2 \\
\end{matrix} \right.$$
}

\kpoint{ 单侧导数}

\end{problem}


\makepart{填空题}

\begin{problem}
设 $y = f\left( \ln x \right)\e^{f\left( x \right)}$,其中 $f$
可微,则 $\dy =$\fillin{
$\displaystyle \e^{f(x)}\left[ \frac{1}{x}f'(\ln x) + f'(x)f(\ln x) \right]$
}.

\kpoint{微分的四则运算}

\end{problem}           

\begin{problem}
设 $\displaystyle f\left( x \right) = \frac{1 - x}{1 + x}$,则
$f^{(n)}\left( x \right) =$ \fillin{$( - 1)^{n}2 \cdot n!(1 + x)^{- (n + 1)}$}.

\kpoint{高阶导数}

\end{problem} 

          
\begin{problem}
	设 $(x_{0},y_{0})$ 是抛物线 $y = {ax}^{2} + bx + c$
上的一点,若在该点的切线过原点,则系数应满足的关系是\fillin{$ax_{0}^{2} = c$,$b$ 任意}.

\kpoint{导函数}

\end{problem}           


\begin{problem}
设 $y = f(x)$,且
$\displaystyle \lim_{h \rightarrow 0}\frac{f\left( x_{0} \right) - f\left( x_{0} + 2h \right)}{6h} = 3$,则
${\dy|}_{x = x_{0}} =$\fillin{$- 9\dx$}.

\kpoint{可导与微分的关系}

\end{problem} 

          
\begin{problem}
设 $\displaystyle f\left( \sqrt{x} \right) = x + \frac{1}{\sqrt{x}}$,则
$\displaystyle f''{\left( x \right)|}_{x = 1}$=\fillin{ 4}.

\kpoint{高阶导数}

\end{problem}           


\begin{problem}
设 $f(x)$ 具有二导数,且
$f'\left( x \right) = \left[ f\left( x \right) \right]^{2}$,则
$f''(x)$=\fillin{$2\left[ f\left( x \right) \right]^{3}$}.

\kpoint{复合函数求导}

\end{problem}           


\begin{problem}
设函数$f\left( x \right) = \left( x + 1 \right)\left( x + 2 \right)\left( x + 3 \right)\ldots(x + n)$
(其中 $n$ 为正整数),则$f'\left( 0 \right) =$\fillin{$\displaystyle {n!}\sum_{{k = 1}}^{{n}}\frac{{1}}{{k}}$}.

\kpoint{对数求导}

\end{problem}           

\begin{problem}
	曲线 $y = \left( 1 + x \right)\e^{x}$ 在点 $x = 0$ 处的切线方程为
$y =$\fillin{$2x + 1$}.

\kpoint{导函数}

\end{problem}           

\begin{problem}
设 $f\left( x \right) = x^{2}$,则
$f'\left[ f\left( x \right) \right] =$\fillin{ $2x^{2}$}.

\kpoint{复合函数导数}

\end{problem}           

\begin{problem}
某商品的需求量 $Q$ 与价格 $P$ 的关系为 $Q = P^{5}$,则需求量
$Q$ 对价格 $P$ 的弹性是\fillin{5}.

\kpoint{弹性}

\end{problem}           

\begin{problem}
	设函数 $f\left( u \right)$ 二阶可导,且
$y = f\left( \ln x \right)$,则 $y'' =$\fillin{$\displaystyle \frac{1}{x^{2}}\left[ f''(\ln x) - f'(\ln x) \right]$}.

\kpoint{复合函数的导数}

\end{problem}           

\begin{problem}
	设 $\displaystyle f(x) = \left( 1 + \frac{1}{x} \right)^{x}$,则
$\displaystyle f'\left( \frac{1}{2} \right) =$\fillin{$\displaystyle \sqrt{3}\left( \ln 3 - \frac{2}{3} \right)$}.

\kpoint{对数求导}

\end{problem}           

\begin{problem}
设函数 $f(x)$ 在 $( - \infty, + \infty)$ 上可导,且
$y = f\left( x^{2006} \right) + \left[ f\left( x \right) \right]^{2006}$,\\
则
$\displaystyle \frac{\dy}{\dx} =$
\fillin{$f'\left( x^{2006} \right) \cdot 2006 \cdot x^{2005} + 2006 \cdot \left[ f(x)\right]^{2005} \cdot f'(x) = 2006\left\{ x^{2005}f'\left( x^{2006} \right) + \left[ f(x)\right]^{2005}f'(x) \right\} $}.

\kpoint{复合函数求导}

\end{problem}           

\begin{problem}设 $\displaystyle \ln\sqrt{x^{2} + y^{2}} = \arctan\frac{y}{x}$,则
$\displaystyle \frac{\dy}{\dx} =$\fillin{$\displaystyle \frac{x + y}{x - y}$}.

\kpoint{隐函数求导}
\end{problem}

\begin{problem}
	设曲线 $f\left( x \right) = x^{n}$ 在点 $(1,1)$ 处的切线与 $x$
	轴的交点为
	$(\xi_{n},0)$,则$\displaystyle \lim_{n \rightarrow \infty}{f(\xi_{n})} =$\fillin{$\e^{- 1}$}.

\anlys{$f' = nx^{n - 1},$ 切线:
$y - 1 = n \cdot \ \ 1^{\left\{ n - 1 \right\}} \cdot \left( x - 1 \right) \Rightarrow y = nx - n + 1$.
切线与 $x$ 轴的交点:$\xi_{n} = \frac{n - 1}{n}$,
$\displaystyle \lim_{n \rightarrow \infty}{f\left( \xi_{n} \right) = \lim_{n \rightarrow \infty}\left( \frac{n - 1}{n} \right)^{n} = \frac{1}{\e}}$}

\kpoint{导函数}

\end{problem}           \begin{problem} 函数 $f\left( x \right) = \sqrt[3]{x}|x|$ 在点 $x = 0$ 处的导数
$f'\left( 0 \right) =$\fillin{0}.

\kpoint{单侧函数}

\end{problem}           \begin{problem}设 $y = 2x + 1$,则其反函数 $x = x(y)$ 的导数
$x'\left( y \right) =$\fillin{$\frac{1}{2}$}.

\kpoint{反函数求导}

%(与13 重复)18.设函数 $f\left( x \right)$ 在
%$( - \infty, + \infty)$上可导,且
%$y = f\left( x^{2006} \right) + \left\left[ f\left( x \right) \right\right]^{2006}$,则$\frac{{dy}}{dx} =$\fillin{$\begin{matrix}
%y' = f'\left( x^{2006} \right) \cdot 2006 \cdot x^{2005} + 2006 \cdot \left[ f(x)\right]^{2005} \cdot f'(x) \\
%= 2006\left\{ x^{2005}f'\left( x^{2006} \right) + \left[ f(x)\right]^{2005}f'(x) \right\} \\
%\end{matrix}$}
%
%\kpoint{复合函数求导}

\end{problem}           

\begin{problem}
设 $ 
\begin{cases}
x = t - \ln(1 + t) \\
y = t^{3} + t^{2} \\
\end{cases}\ $,则
$\displaystyle \frac{\d^{2}y}{\dx^{2}} =$\fillin{$\displaystyle \frac{(6t + 5)(t + 1)}{t}$}

\anlys{
$\displaystyle \frac{\dy}{\dx} = \frac{3t^{2} + 2t}{1 - \frac{1}{1 + t}} = 3t^{2} + 5t + 2$.}

\kpoint{参数方程求导数}

\end{problem}           


\begin{problem}
	问自然数 $n$ 至少多大,才能使 $f(x) = \begin{cases}
x^{n}\sin\displaystyle \frac{1}{x},\quad   &x \neq 0\\
0,\quad   &x = 0 
\end{cases}$ 在$x=0$处二阶可导 ($f''\left( 0 \right)$存在),并求其值.\fillin{$n \geq 4$}\fillin{$f''\left( 0 \right) = 0$}.

\anlys{由导数的定义可得:
	\begin{equation*}
	\begin{split}
	f'(0) &= \lim_{x \rightarrow 0}\mspace{2mu}\frac{x^{n}\sin\frac{1}{x} - 0}{x - 0} = \lim_{x \rightarrow 0}\mspace{2mu} x^{n - 1}\sin\frac{1}{x} = 0(n \geq 2),f' = nx^{n - 1}\sin\frac{1}{x} - x^{n - 2}\cos\frac{1}{x},\\
f''(0) &= \lim_{x \rightarrow 0}\mspace{2mu}\frac{nx^{n - 1}\sin\frac{1}{x} - x^{n - 2}\cos\frac{1}{x} - 0}{x - 0} = \lim_{x \rightarrow 0}\mspace{2mu}\left( nx^{n - 2}\sin\frac{1}{x} - x^{n - 3}\cos\frac{1}{x} \right). 
\end{split}
\end{equation*}
故当$n\geq 4$时$f''(0)=0$.}

\kpoint{高阶导数}

\end{problem}


\makepart{计算题}


\begin{problem} 设函数 $f(x) = \begin{cases}
3x + 2,\ \ \ \  x \leq 0 \\
\e^{x} + 1,\ \ \ \  x > 0 \\
\end{cases} $ , 求 $f'(x)$.

\begin{solution}
	$f'(x) =  \begin{cases}
3,\ \ \ \  x < 0 \\
e^{x},\ \ \ \  x > 0 \\
\end{cases},$ (在 $x = 0$ 处不可导)
\end{solution}

\kpoint{单侧导数}
\end{problem}

\begin{problem} 设 $\displaystyle y = \frac{x\arctan x}{1 + x}$,求 ${\dy}$.

\begin{solution}
	$\displaystyle \dy = \frac{\arctan x + \frac{x^2+x}{1 + x^{2}}}{(1 + x)^{2}}\dx$.
\end{solution}

\kpoint{可导与微分的关系}
\end{problem}

\begin{problem} 设 $y=3^{x} + x^{3} + x^{\cos 3x}$, 求 $y'$.

\begin{solution}
$\displaystyle 3^{x}\ln 3 + 3x^{2} + x^{\cos 3x}\left( - 3\sin 3x\ln x + \frac{\cos 3x}{x} \right)$.
\end{solution}
\kpoint{导数的四则运算}
\end{problem}

\begin{problem} 
	设 $y = y(x)$ 由方程 $y = f\left[ x + g\left( y \right)\right]$.
所确定,其中 $f$ 和 $g$ 均可导,求 $y'$.

\begin{solution}
$\displaystyle \frac{f'\left[ x + g\left( y \right)\right]}{1 - f'\left[ x + g(y)\right] \cdot g'(y)\ }$.
\end{solution}

\kpoint{隐函数求导}
\end{problem}

\begin{problem}
	 设 $y = x \cdot \arctan\frac{1}{x} + \ln{\sqrt{1 + x^{2}}\ }$,求
$y'$.

\begin{solution}
 $\displaystyle \arctan\frac{1}{x}$.
\end{solution}

\kpoint{导数的四则运算}
\end{problem}

\begin{problem}
	设
$\displaystyle y = \frac{(x + 1)^{2}\sqrt[4]{x - 2}}{\sqrt[3]{(x + 2)^{2}}}$,求
$y'$.

\begin{solution}
$\displaystyle \frac{(x + 1)^{2}\sqrt[4]{x - 2}}{\sqrt[3]{(x + 2)^{2}}}\left[ \frac{2}{x + 1} + \frac{1}{4(x - 2)} - \frac{2}{3(x + 2)} \right]$.
\end{solution}

\kpoint{导数的四则运算}
\end{problem}

\begin{problem}
	已知 $\displaystyle y^{x} = x^{y}$,求 $y'$.

\begin{solution}
	$\displaystyle \frac{y(y - xln\ y)}{x(x - yln\ x)}$.
\end{solution}

\kpoint{隐函数求导}
\end{problem}

\begin{problem}
	由 $\e^{x^{2} + y^{2}} + \sin\left( {xy} \right) = 5$ 确定 $y$ 是
$x$ 的函数 $y(x)$,求 $y'(x)$.

\begin{solution}
$\displaystyle y' = - \frac{2x\e^{x^{2} + y^{2}} + y\cos(xy)}{2y\e^{x^{2} + y^{2}} + x\cos(xy)}$
\end{solution}

\kpoint{隐函数求导}
\end{problem}

\begin{problem}
	函数 $y = y(x)$ 由方程 $\e^{x} - \e^{y} - xy = 0$ 确定,求
$\left. \ \frac{\d^{2}y}{\dx^{2}} \right|_{x = 0}$

\begin{solution}
	对方程两边关于 $x$ 求导,得
$$\e^{x} - \e^{y}y' - y - xy' = 0,$$

两边关于 $x$ 再求导,得

$$\e^{x} - \e^{y}y^{'2} - \e^{y}y'' - y' - y' - xy'' = 0$$

又当 $x = 0$ 时,$y = 0$,于是 $y'(0) = 1$,故
$\displaystyle \left. \ \frac{\d^{2}y}{\dx^{2}} \right|_{x = 0} = - 2$.
\end{solution}

\kpoint{隐函数求导}
\end{problem}

\begin{problem}
	设 $\begin{cases}
x = 2\sin 3t \\
y = \e^{t} + \ln 2 \\
\end{cases},$ 求 $\displaystyle \frac{\d^{2}y}{\dx^{2}}$.

\begin{solution}
   $\displaystyle \frac{\e^{t}(\cos 3t + 3\sin 3t)}{36\cos^{3}3t}$.
\end{solution}

\kpoint{参数方程求导}
\end{problem}

\begin{problem}
	设曲线方程为 $ \begin{cases}
x = t + \sin t + 2 \\
y = t + \cos t \\
\end{cases}, $求此曲线在点 $x = 2$
处的切线方程,及$\displaystyle \frac{\d^{2}y}{\dx^{2}}$.

\begin{solution}
当 $x = 2$
时,$t = 0$,$y = 1$,$\displaystyle \frac{{\dy}}{\dx} = \frac{1 - \sin t}{1 + \cos t}$,
$\displaystyle \left. \ \frac{{\dy}}\dx \right|_{t = 0} = \frac{1}{2}$,

切线方程:$$y - 1 = \frac{1}{2}\left( x - 2 \right);$$
二阶导数:
$$\frac{\d^{2}y}{\dx^{2}} = \frac{\d}{{\dt}}\left( \frac{{\dy}}{\dx} \right) \cdot \frac{1}{\displaystyle \frac{\dx}{{\dt}}} = \frac{\sin t - \cos t - 1}{(1 + \cos t)^{3}}$$
\end{solution}

\kpoint{高阶导数}
\end{problem}

\begin{problem} 设 $f(x)$ 存在二阶连续导数,且
$\displaystyle \lim_{x \rightarrow 0}\mspace{2mu}\frac{f\left( x \right)}{x} = 0$,
$f''\left( 0 \right) = 4$, 求
$\displaystyle \lim_{x \rightarrow 0}\mspace{2mu}\left( 1 + \frac{f(x)}{x} \right)^{\frac{1}{x}}$.

\begin{solution}
	$\e^2$.
\end{solution}

\kpoint{导函数}
\end{problem}

\begin{problem}
	设曲线 $f(x)$在 $\left[ 0,1\right]$ 上可导,且
	$y = f\left( \sin^{2}x \right) + f(\cos^{2}x)$,求
	$\displaystyle \frac{{\dy}}{\dx}$


\begin{solution}
$y' = \left[ f'(\sin^{2}x) - f'\left( \cos^{2}x \right) \right]\sin{2x}$
\end{solution}

\kpoint{复合函数求导}

\end{problem}


\makepart{综合与应用题}

\begin{problem}一人以2m每秒的速度通过一座高20m的桥,此人的正下方有一小船以$\displaystyle \frac{4}{3}$
m每秒的速度与桥垂直的方向前进,求第5秒末人与船相离的速率。

\begin{solution}
	设在时刻$t$人与船的距离为$s$,则

$$
s = \sqrt{20^{2} + (2t)^{2} + \left( \frac{4}{3}t \right)^{2}} = \frac{1}{3}\sqrt{3600 + 52t^{2}}, $$
$$\frac{{\ds}}{{\dt}} = \left. \ \frac{52}{3}\frac{t}{\sqrt{3600 + 5t^{2}}},\qquad \frac{{\ds}}{{\dt}} \right|_{t = 5} = \frac{26}{21}(\text{m/s}).
$$

答:第5秒末人与船相离的速率为$\displaystyle \frac{26}{21}$(m/s).
\end{solution}

\kpoint{ 导函数}

\end{problem}           

\begin{problem} 设 $f(x) = \begin{cases}
k + \ln(1 + x) & x \geq 0 \\
e^{\sin x} & x < 0 \\
\end{cases} $,当 $k$ 为何值时,点 $x = 0$
处可导;此时求出 $f'(x)$.

\begin{solution}
	当 $k = 1$ 时,$f(x)$ 在点 $x = 0$ 处可导;此时
$f'(x) = \begin{cases}
\displaystyle \frac{1}{1 + x} & x \geq 0 \\
\e^{\sin x}\cos x & x < 0 \\
\end{cases} $.
\end{solution}

\kpoint{ 单侧导数}

\end{problem}           

\begin{problem} 
若 $y = f(x)$是奇函数且在点 $x = 0$ 处可导,则点 $x = 0$ 是函数
$\displaystyle F\left( x \right) = \frac{f\left( x \right)}{x}$什么类型的间断点?说明理由.

\begin{solution}
	由 $f(x)$ 是奇函数,且在点 $x = 0$ 处可导,知 $f(x)$在点
$x = 0$ 处连续,$f\left( 0 \right) = - f(0)$,则
$f\left( 0 \right) = 0$,于是
$\displaystyle \lim_{x \rightarrow 0}\mspace{2mu} F(x) = \lim_{x \rightarrow 0}\mspace{2mu}\frac{f(x) - f(0)}{x - 0} = f'(0)$
存在,
故点 $x = 0$ 是函数 $F\left( x \right)$ 第一类间断点(可去).
\end{solution}

\kpoint{ 可导与连续的关系}
\end{problem}

\begin{problem}
试确定常数 $a,b$ 的值,使得函数
$f\left( x \right) = \begin{cases}
2e^{x} + a & x < 0 \\
x^{2} + bx + 1 & x \geq 0 \\
\end{cases}$ 处处可导.

\begin{solution}
	为使 $f\left( x \right)$ 在点 $x = 0$ 处连续,必须
$$\lim_{x \rightarrow 0^{-}}{\,f}\left( x \right) = \lim_{x \rightarrow 0^{+}}{\,f}\left( x \right) = f(0),$$
即
$$\lim_{x \rightarrow 0^{-}}{\,f}\left( x \right) = 2 + a,\ \lim_{x \rightarrow 0^{+}}{\,f}\left( x \right) = f\left( 0 \right) = 1,$$ 所以
$a = - 1$.

为使 $f\left( x \right)$ 在点 $x = 0$ 处可导,必须
$f_{-}'\left( 0 \right) = f_{+}'(0)$,而
\begin{equation*}\begin{split}
f_{-}'(0) = \lim_{x \rightarrow 0^{-}}\mspace{2mu}\frac{f(x) - f(0)}{x - 0} = \lim_{x \rightarrow 0^{-}}\mspace{2mu}\frac{2\left( e^{x} - 1 \right)}{x} = 2, \\
f_{+}'(0) = \lim_{x \rightarrow 0^{+}}\mspace{2mu}\frac{f(x) - f(0)}{x - 0} = \lim_{x \rightarrow 0^{+}}\mspace{2mu}\frac{x^{2} + bx}{x} = b, \\
\end{split}
\end{equation*} 
所以 $b = 2$.
\end{solution}

\kpoint{ 单侧导数}

(导入时去掉了表格 )

\end{problem}

\begin{problem}
已知某商品的需函数为
$\displaystyle Q = \frac{1200}{P}$,试求:

(1) 从 $P = 30$ 到 $P = 20,25,32,50$ 各点间的需求弹性;

(2) $P = 30$ 时的需求弹性,并说明其经济意义。

\begin{solution}
	\begin{table}[htbp]
		\begin{center}
		\begin{tabular}{|c|c|c|c|c|c|}
			\hline
			$P$ & 20 & 25 & 30 & 32 & 50 \\
			\hline
			$Q$ & 60 & 48 & 40 & 37.5& 24\\
			\hline
			$\Delta P$ & -10& -5 & & 2& 20\\
			\hline
			$\Delta Q$ & 20 & 8 & &-2.5 &-16\\
			\hline
			$\displaystyle \frac{\Delta P}{P}$& -1/3 & -1/6 & & 1/15& 2/3\\
			\hline
			$\displaystyle \frac{\Delta Q}{Q}$ & 0.5 & 0.2 & & -0.0625& -0.4\\
			\hline
			$\bar{\eta}$ & 1.5 & 1.2 & & 15/16 & 0.6\\
			\hline 
		\end{tabular}
\end{center}
	\end{table}
$\displaystyle \eta(P) = - Q'(P)\frac{P}{Q(P)} = 1 \Rightarrow \eta(30) = 1.$

经济意义:在 $P = 30$ 时,价格上涨 1\%,则需求减少
1\%;而价格下跌1\%,则需求增加 1\%.
\end{solution}



\kpoint{ 弹性}

\end{problem}

\begin{problem}
	设 $f\left( x \right)$ 对任何 $x$ 满足
	$f\left( x + 1 \right) = 2\ f(x)$, 且 $f\left( 0 \right) = 1$,
	$f'(0) = C$ (常数),求 $f'(1)$.

\begin{solution}
注意题设 $f\left( x \right)$ 仅在 $x = 0$ 的导数存在. 故函数在其它点的导数应用导数定义求之.

令 $x = 0$ 有 $f(1) = 2f\left( 0 \right) = 2$, 所以

\begin{equation*}
\begin{split}
f'(1) &= \lim_{\Delta x \rightarrow 0}\mspace{2mu}\frac{f(1 + \Delta x) - f(1)}{\Delta x} = \lim_{x \rightarrow 0}\mspace{2mu}\frac{2f(x) - 2}{x} \\
&= 2\lim_{x \rightarrow 0}\mspace{2mu}\frac{f(x) - f(0)}{x} \\
&= 2f'(0) = 2C \\
\end{split}
\end{equation*}
\end{solution}

\kpoint{ 导数概念}

\end{problem}          


\begin{problem}试确定常数 $a,b$ 的值, 使函数
$f\left( x \right) =  \begin{cases}
\cos{3x} & x \leq 0 \\
be^{x} + a & x > 0 \\
\end{cases} $ , 在 $x = 0$
处可导.

\begin{solution}
由	
$$\lim_{x \rightarrow 0}{,f}\left( x \right) = \lim_{x \rightarrow 0^{+}}\mspace{2mu} f\left( x \right) = f(0),$$
得 $a + b = 1$.

又
$$\lim_{x \rightarrow 0^{-}}\mspace{2mu}\frac{f(x) - f(0)}{x} = \lim_{x \rightarrow 0^{-}}\mspace{2mu}\frac{\cos 3x - 1}{x} = 0,$$

$$\lim_{x \rightarrow 0^{+}}\mspace{2mu}\frac{f(x) - f(0)}{x} = \lim_{x \rightarrow 0^{+}}\mspace{2mu}\frac{be^{x} + a - 1}{x} = \frac{b\left( e^{x} - 1 \right)}{x} = b,$$

所以 $a = 1,\ b = 0$ 时 $f\left( x \right)$ 在 $x = 0$ 处可导.
\end{solution}

\kpoint{ 单侧导数}

\end{problem}           

\begin{problem} 设 $f\left( x \right) =  \begin{cases}
\cos x & x \leq 0 \\
ax^{2} + bx + c & x > 0 \\
\end{cases}  $ , 求 $a,b,c$ 的值, 使 $f\left( x \right)$
在 $x = 0$ 处二阶可导.

\begin{solution}
	首先在 $x = 0$ 连续,故
$$\lim_{x \rightarrow 0^{+}}{\,f}\left( x \right) = \lim_{x \rightarrow 0^{-}}{\ f}\left( x \right)\mspace{2mu} = f(0),$$
得 $c = 1$.

在 $x = 0$ 处一阶可导,
故
$$\begin{matrix}
\displaystyle \lim_{x \rightarrow 0^{-}}\mspace{2mu}\frac{f(x) - f(0)}{x} = \lim_{x \rightarrow 0^{-}}\mspace{2mu}\frac{\cos x - 1}{x} = 0 \\
\displaystyle \lim_{x \rightarrow 0^{+}}\mspace{2mu}\frac{f(x) - f(0)}{x} = \lim_{x \rightarrow 0^{+}}\mspace{2mu}\frac{ax^{2} + bx}{x} = b 
\end{matrix}$$
得 $b = 0$. 
所以
$$f'(x) =\begin{cases}
-\sin x, \ &x\leq 0\\
2ax, \ &x>0
\end{cases} .$$

在 $x = 0$ 处二阶可导

$$\begin{matrix}
\displaystyle \lim_{x \rightarrow 0^{-}}\mspace{2mu}\frac{f'(x) - f'(0)}{x} = \lim_{x \rightarrow 0^{-}}\mspace{2mu}\frac{- \sin x}{x} = - 1, \\
\displaystyle \lim_{x \rightarrow 0^{+}}\mspace{2mu}\frac{f'(x) - f'(0)}{x} = \lim_{x \rightarrow 0^{+}}\mspace{2mu}\frac{2ax}{x} = 2a.
\end{matrix}$$

得到 $2a = - 1$, 即 $\displaystyle a = - \frac{1}{2}$.
\end{solution}

\kpoint{ 单侧导数}

\end{problem}
\makepart{分析与证明题}

\begin{problem}
	设函数 $f(x)$ 在 $( - \infty, + \infty)$ 上有定义,对任意的
	$x,y \in ( - \infty, + \infty)$ 有
	$f\left( x + y \right) = {\ f}\left( x \right) + f\left( y \right) + xy$,且
	$f'(0) = 1$,证明 $f'\left( x \right) = 1 + x$
	
\begin{solution}
取 $x = y = 0$,得 $f(0) = 0$;又取 $y = {\Delta x}$, 得
$$\Delta y = f(x + \Delta x) - f(x) = f(\Delta x) + x\Delta x.$$
故
$$
f'(x) = \lim_{\Delta x \rightarrow 0}\frac{\Delta y}{\Delta x} = \lim_{\Delta x \rightarrow 0}\frac{f(\Delta x)}{\Delta x} + x = \lim_{\Delta x \rightarrow 0}\frac{f(0 + \Delta x) - f(0)}{\Delta x} + x = f'(0) + x = 1 + x.$$
\end{solution}

\kpoint{ 导函数}
\end{problem}

\begin{problem}
设$f\left( x \right) = g\left( x \right)\sin^{\alpha}{(x - x_{0})}\ \ (\alpha > 1)$,其中
	$g\left( x \right)$ 在 $x_{0}$ 处连续,证明:
	$f\left( x \right)$在 $x_{0}$ 处可导。
	
\begin{solution}
	因为
	\begin{equation*}
	\begin{split}
	\lim_{x \rightarrow x_{0}}\frac{f(x) - f\left( x_{0} \right)}{x - x_{0}} &= \lim_{x \rightarrow x_{0}}\frac{g(x)\sin^{\alpha}\left( x - x_{0} \right)}{x - x_{0}}\\
	& = \lim_{x \rightarrow x_{0}}\left\{ \left\lbrack g(x)\sin^{\alpha - 1}\left( x - x_{0} \right) \right\rbrack \cdot \frac{\sin\left( x - x_{0} \right)}{x - x_{0}} \right\} = \left\{ \begin{matrix}
g\left( x_{0} \right) & \alpha = 1 \\
0 & \alpha > 1 \\
\end{matrix} \right.\ \ 
\end{split}
\end{equation*}
所以 $f\left( x \right)$ 在 $x_{0}$ 处可导
\end{solution}

\kpoint{ 导数概念}
\end{problem}

\begin{problem}设 $f(x)$ 在 $( - \infty, + \infty)$ 上有定义且在 $x = 0$
处连续,对任意的 $x_{1},\ \ x_{2}$ 均有
	$$f\left( x_{1} + x_{2} \right) = {\ f}\left( x_{1} \right) + f\left( x_{2} \right).$$

(1) 证明 $f\left( x \right)$ 在 $( - \infty, + \infty)$ 上连续;

(2) 又设 $f'\left( 0 \right) = a$ (常数), 证明
$f\left( x \right) = ax$.

\begin{solution}
(1) 考虑
$\displaystyle \lim_{\Delta x \rightarrow 0}\mspace{2mu}\Delta y$.
令 $x_{1} = 0$, $x_{2} = 0$,得 $f\left( 0 \right) = 0$,
又令 $x_{1} = x$,$x_{2} = \Delta x$,则
$$f\left( x + \Delta x \right) = f\left( x \right) + f(\Delta x),$$
即 $$\Delta y = f\left( x + \Delta x \right) - f(x) = f(\Delta x),$$
而
$f\left( x \right)$ 在点 $x = 0$ 处连续,所以

$$\lim_{\Delta x \rightarrow 0}\mspace{2mu}\Delta y = \lim_{\Delta x \rightarrow 0}f\left( {\Delta x} \right) = f\left( 0 \right) = 0$$

故 $f(x)$ 在 $( - \infty, + \infty)$ 内连续。

(2) 考虑
$\displaystyle f'(x) = \lim_{\Delta x \rightarrow 0}\frac{\Delta y}{\Delta x}$.
对 $\forall x \in \left( - \infty, + \infty \right),$ 令
$x_{1} = x$,$x_{2} = \Delta x$,
则 
$$f\left( x + \Delta x \right) = f\left( x \right) + f(\Delta x),$$
即 $$\Delta y = f\left( x + \Delta x \right) - f(x) = f(\Delta x)$$
所以
$$f'(x) = \lim_{\Delta x \rightarrow 0}\mspace{2mu}\frac{\Delta y}{\Delta x} = \lim_{\Delta x \rightarrow 0}\mspace{2mu}\frac{f(\Delta x)}{\Delta x} = f'(0) = a,$$
得
$f\left( x \right) = ax + c$. 因为 $f\left( 0 \right) = 0$,
故 $f\left( x \right) = ax$.
\end{solution}

\kpoint{ 导函数}

\end{problem}          

\begin{problem} 设函数 $f(x)$ 对任何实数 $x_{1},\ \ x_{2}$ 有
$f\left( x_{1} + x_{2} \right) = {\ f}\left( x_{1} \right) + f\left( x_{2} \right)$.
且 $f'\left( 0 \right) = 1$, 证明:函数 $f(x)$ 可导,且
$f'\left( x \right) = 1$.

\begin{solution}
由
$f\left( 0 + 0 \right) = {\ f}\left( 0 \right) + f\left( 0 \right)\ {\ } \Rightarrow \ \ {\ \ f}\left( 0 \right) = 0$.
所以对任何实数 $x$ 有
\begin{equation*}\begin{split}
f'(x) &= \lim_{\Delta x \rightarrow 0}\mspace{2mu}\frac{f(x + \Delta x) - f(x)}{\Delta x} \\
&= \lim_{Ax \rightarrow 0}\mspace{2mu}\frac{f(x) + f(\Delta x) - f(x)}{\Delta x} = \lim_{\Lambda x \rightarrow 0}\mspace{2mu}\frac{f(\Delta x)}{\Delta x} \\
&= \lim_{\Delta x \rightarrow 0}\mspace{2mu}\frac{f(\Delta x) - f(0)}{\Delta x} = f'(0) = 1.
\end{split}
\end{equation*}
\end{solution}

\kpoint{ 导函数}
\end{problem}

\begin{problem}
设函数 $\displaystyle f\left( x \right) = \frac{\sqrt{{x\ }}}{\sqrt{1 + x} + 1}$,
证明 $f\left( x \right)$ 在 $x = 0$ 处右连续, 但右导数不存在.

\begin{solution}
	易知
$$ \lim_{x \rightarrow 0^{+}}\mspace{2mu} f(x) = \lim_{x \rightarrow 0^{+}}\mspace{2mu}\frac{\sqrt{x}}{\sqrt{1 + x} + 1} = 0 = f(0),$$
所以 $f\left( x \right)$ 在 $x = 0$ 处右连续.

又
$$f_{+}'(0) = \lim_{x \rightarrow 0^{+}}\mspace{2mu}\frac{f(x) - f(0)}{x - 0} = \lim_{x \rightarrow 0^{+}}\mspace{2mu}\frac{\sqrt{x}}{x(\sqrt{1 + x} + 1)}\ \text{不存在},$$
故函数在 $x = 0$ 处右导数不存在.
\end{solution}

\kpoint{ 单侧导数}
\end{problem}